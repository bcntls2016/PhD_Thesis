

%		Introduce superfluids and their properties
%		* Frictionless capillary flow (no viscosity)
%		* Creeping up the walls seemingly defying gravity
%		* capillary fountain
%		* second sound: heat propagating in waves with a constant speed of "second sound" instead of diffusion
%		* Fluid is irrotational
%		* Vorticity is quantised
	 

%%		E, 1908, July 10:		Kamerlingh Onnes, Liquefaction of helium-4,\\
%%		E, 1932, July:			John C. McLennan, Observed liquid helium stops boiling
%below 2.2 K. \\
%%		E, 1932:				W.H. Keesom, A.P. Keesom, observed a singularity in the specific
%heat at T=2.2 K and called it the lambda-temperature, because of the shape of
%the temperature dependence of the specific heat resembling the greek letter
%lambda.\\
%%		E, 1935, February 16:	E.F. Burton, measured sharply decreasing viscosity
%below 2.2 K\\
%%		T, 1935, August 16:		F. London, found that the magnitude of the zero-point
%energy of helium-4 is comparable to the Van der Waals interaction. This
%explains why helium doesn't freeze at T=0 at normal atmospheric pressure.\\
%%		E, 1936, May:			W.H. Keesom, A.P. Keesom, Observed abnormally thermal
%conductance, calling it 'supra-heat-conducting'\\
%%		E, 1937, July 10:		J.F. Allen,R. Peierls, M. Zaki Uddin, also observed
%abnormally high heat conductance. This was the reason for the liquid not
%boiling below 2.2 K\\
%%		E, 1937, December 3:	Kapitza, observed that below lambda-point viscosity of
%helium II is roughly 1500 smaller than helium I at normal pressure. In analogy
%with superconductors he concluded that helium below lambda enters a special
%state which he called superfluid.	This was the first mention of the word
%superfluid.\\
%%		E, 1937, December 22:	Allen and Misener (1938) discovered that helium-ii is
%not just a liquid with a very low viscosity, but that its hydrodynamics
%required a completely new interpretation.\\
%%		E, 1938, 05 February: 	J.F. Allen, discovery of fountain effect\\
%%		T, 1938, April 9:		F. London, connects behaviour of helium-II to BEC (ideal
%BE-gas). calculated Tc=3.09 K and Cv(t) for ideal BE-gas and they were very
%close to helium-II. He concluded that it was difficult not to imagine a
%connection to BEC\\
%%		T, 1938, May 21:		L. Tisza, birth of the 2-fluid model\\
%
%					which leads to 
%
%	Explanation of superfluidity by 
%		* Fritz London (April 1938, London, F., Nature, 141, 643 (1938))
%l-transition in liquid helium is analogous to Bose-Einstein condensation
%%		* Laszlo Tisza (May 1938, L. Tisza, Nature, 141, 913 (1938)) extended
%London's proposal by invoking a two-fluid model for helium II, which could
%qualitatively explain the observed transport phenomena, including the fountain
%effect.
%%		* Lev Landau (1941) if the spectrum of elementary excitations satisfies
%suitable criteria, the flow of of the fluid cannot dissipate energy. Postulated
%"phonons" and "rotons" and the famous Landau criterion for superfluidity
%%		* Nikolay Nikolayevich Bogolyubov (Oct. 12, 1946. Publ. 1947): Derivation of
%the elementary excitation spectrum from a molecular theory, making no
%assumptions about the structure of the energy spectrum.


%	\section{Random stuff about BEC}
%	Most general (time-independent) many-body Hamiltonian
%	\begin{align}
%		H(\vec{r}_1,\ldots,\vec{r}_N, \vec{p}_1,\ldots,\vec{p}_N) =
%\sum_{i=1}^N
%\qty(-\frac{\hbar^2}{2m_i}\nabla_{\vec{r}_i}^2)+V(\vec{r}_i,\ldots,\vec{r}_N)
%	\end{align}
%	And accompanying Schrödinger equation to solve
%	\begin{align}
%		i\hbar\frac{\partial}{\partial t}\Psi(\vec{r}_1,\ldots,\vec{r}_N,t) =
%\qty[\sum_{i=1}^N
%\qty(-\frac{\hbar^2}{2m_i}\nabla_{\vec{r}_i}^2)+V(\vec{r}_i,\ldots,\vec{r}_N)
%]\Psi(\vec{r}_1,\ldots,\vec{r}_N,t)
%	\end{align}
%For a 2-body system and a potential that only depends on the relative
%coordinate $\vec{r}_1-\vec{r}_2$ the Hamiltonian reduces reduces to
%	\begin{align}
%		H(\vec{r}_1,\vec{r}_2) =
%-\qty(\frac{\hbar^2}{2m_1}\nabla_{\vec{r}_1}^2+\frac{\hbar^2}{2m_2}\nabla_{\vec
%{r}_2}^2)+V(\vec{r}_1-\vec{r}_2)
%	\end{align}
%By either introducing the relative coordinate $\vec{r}$ and the center-of-mass
%(CM) coordinate $\vec{R}$, or the relative momentum
%$\vec{p}$ and the total momentum $\vec{P}$, we can consider the motion of the
%CM itself ($\vec{R}$) and the motion relative to the CM
%	($\vec{r}$):
%	\begin{align}
%		H(\vec{R},\vec{r}) =
%-\qty(\frac{\hbar^2}{2M}\nabla_{\vec{R}}^2+\frac{\hbar^2}{2\mu}\nabla_{\vec{r}}
%^2)+V(\vec{r}),
%	\end{align}
%with $M=m_1+m_2$ and $\mu=\frac{m_1m_2}{m_1+m_2}$ (reduced mass of the system).
%	Schrödinger equation to solve
%	\begin{align}
%i\hbar\frac{\partial}{\partial t}\Psi(\vec{R},\vec{r},t) &=
%H(\vec{R},\vec{r})\Psi(\vec{R},\vec{r},t) \\
%&=
%\qty[-\qty(\frac{\hbar^2}{2M}\nabla_{\vec{R}}^2+\frac{\hbar^2}{2\mu}\nabla_
%{\vec{r}}^2)+V(\vec{r})]
%		\Psi(\vec{R},\vec{r},t) \label{eq:2b-se}
%	\end{align}
%	Assuming a separable solution and the given time-independence of the potential
%	\begin{align}
%\Psi(\vec{R},\vec{r},t) =
%\Phi(\vec{R})\psi(\vec{r})\mathrm{e}^{-i(E_{CM}+E)t/\hbar}
%	\end{align}
%and the 2-body Schrödinger equation (\ref{eq:2b-se}) separates into two
%mutually time-independent ODEs
%	\begin{align}
%-\frac{\hbar^2}{2M}\nabla_{\vec{R}}^2\Phi(\vec{R}) &= E_{CM}\Phi(\vec{R})
%\label{eq:se-fp}\\
%\qty[-\frac{\hbar^2}{2\mu}\nabla_{\vec{r}}^2 + V(\vec{r})]\psi(\vec{r}) &=
%E\psi(\vec{r}). \label{eq:se-muv}
%	\end{align}
%ODE (\ref{eq:se-fp}) describes the center of mass motion for a free particle
%with mass $M$ and energy
%$E_{CM}$. In one dimension the time-\emph{dependent normalizable} solution for
%such a particle,
%initially localized between $-a$ and $a$ is an integral over plane waves over
%all frequencies
%	\begin{align}
%\Phi(X,t) =
%\frac{1}{\pi}\sqrt{\frac{a}{2}}\int_{-\infty}^{+\infty}\!\mathrm{sinc}(ka)\,
%		\exp[i\qty(kX-\frac{\hbar k^2}{2M}t)]\,\mathrm{d}k.
%	\end{align}
%ODE (\ref{eq:se-muv}) describes the motion relative to the CM of a particle of
%reduced mass $\mu$ in a potential $V(\vec{r})$.
%	For hydrogen this is the Coulomb potential. Explicitly
%	\begin{align}
%\qty[-\frac{\hbar^2}{2\mu}\nabla_{\vec{r}}^2 -
%\frac{e^2}{(4\pi\varepsilon_0)r}]\psi(\vec{r}) &= E\psi(\vec{r}),
%	\end{align}
%where it is customary to switch to spherical polar coordinates and separate the
%solutions again
%	\begin{align}
%\psi(\vec{r}) \rightarrow \psi_{E,l,m}(r,\theta,\phi) =
%R_{E,l}(r)Y_{ml}(\theta,\phi),
%	\end{align}
%where $R_{E,l}$ is the radial wave function, fully determined by the energy $E$
%and the orbital angular momentum quantum
%number $l$ and $Y_{ml}$ are the spherical harmonics, fully determined by $l$
%and the magnetic quantum number $m$.
%	
%	\section{Helium-4, a 3-body system with 2 electrons}
%The unperturbed (omit electron-electron repulsion) ground state (as obtained by
%the `single particle model')
%	\begin{align}
%\psi_0^{(0)}(r_1,r_2) &= \psi_{1\mathrm{s}(r_1)}\psi_{1\mathrm{s}(r_2)} \otimes
%		\frac{1}{\sqrt{2}}\qty(\ket{\uparrow\downarrow}-\ket{\downarrow\uparrow}) \\ 
%&= \frac{8}{\pi}\exp[-2(r_1+r_2)] \otimes
%\frac{1}{\sqrt{2}}\qty(\ket{\uparrow\downarrow}-\ket{\downarrow\uparrow})
%	\end{align}
%Or in the `central field approximation' with effective nuclear charge
%$Z_e=1.70$
%	\begin{align}
%\psi_0(r_1,r_2) =
%\frac{4.913}{\pi}\exp[-1.70(r_1+r_2)]\otimes\frac{1}{\sqrt{2}}\qty(\ket
%{\uparrow\downarrow}-\ket{\downarrow\uparrow})
%	\end{align}
