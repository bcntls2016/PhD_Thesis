\newcommand{\nhcomm}[1]{\textit{\textcolor{blue}{(#1)}$_{N}$}}
\newcommand{\nhrepl}[2]{{\footnotesize[#1]}$_{N}$\textbf{#2 ---}}
\newcommand{\nhdel}[1]{{\footnotesize[\textcolor{red}{#1}]}$_{N}$}
\newcommand{\nhadd}[1]{\textbf{\textcolor{green}{#1 ---}}}


\chapter{To conclude}

\lettrine[lines=4]{\color{activeColor}I}{n} this thesis we have explored two complementary aspects of the dynamics of atomic dopants interacting with helium nanodroplets: the fate of a photo-excited alkali on the droplet surface, and the doping process itself. Photo-excited alkalis have been studied in a joint theoretical and experimental effort, whereas the atom-helium droplet collisions leading to doping have been studied only theoretically, but with the view of proposing some fundamental explanations for basic steps in the remarkable experiments leading to the formation of filament shaped nanostructures of atomic cluster.

As a theoretical tool we have made extensive use of the $^4$He-DFT approach, in its static version to obtain equilibrium properties and in its time-dependent version ($^4$He-TDDFT) to simulate excited state dynamics and collisions, as explained in Chapter~\ref{sec:dft-method}. We have applied it to study helium nanodroplets of several thousand atoms (routinely 1000) interacting with atomic dopants. As shown in the following results chapters, this approach has provided deep insight about all the processes studied, as well as quantitative agreement in most of the simulations when comparison with experimental data was possible. This agreement is so reliable that in the few cases where it was not the case, it lead to the identification of a process that had not been taken into account thus far.

The first part of the results was dedicated to excited-state dynamics of an alkali atom bound to the helium droplet surface. In Chapter~\ref{ch:exc-state-dyn} we have studied the dynamics of femtosecond pump-probe photoionisation of Rb atoms attached to He nanodroplets in a combined theory-experiment investigation. We have concluded that it was governed by the competition between the repulsive interaction of the droplet with the Rb atom in an excited state and the attractive interaction of the droplet with the Rb$^+$ cation. Depending on the time delay between the pump and the probe laser pulse, this caused either desorption of Rb* off the droplet or submersion of the Rb$^+$ cation into the droplet interior. We have been able to determine the critical time delay separating these two behaviours and obtained an excellent agreement with experiments, except in the case of the 5p$\,^2\Pi_{3/2}$ excitation of Rb where theory found a surface bound Rb*(5p$\,^2\Pi_{3/2}$)-He exciplex whereas experiment observed its dissociation from the droplet. This agreement allowed us to characterise the desorption dynamics as impulsive with a time scale of $\sim$1~ps for the 6p-excited states and as intermediate between impulsive dissociation and statistical desorption with a time scale of $\sim$100~ps for the 5p-excited states. This interplay between opposing trends (He-Rb* repulsion, He-Rb$^+$ attraction) will be present in other types of clusters and condensed phase systems probed by time-resolved photoionisation spectroscopy. Hence $^4$He-TDDFT simulations will be helpful in interpreting the experimental results in many cases.

In Chapter~\ref{ch:rbhe-exciplexe} we have gone back to understanding the disagreement observed between theory and experiment for Rb*(5p$\,^2\Pi_{3/2}$) excitation. The dynamics of the lowest three excited states (5p) of the Rb–He droplet complex have been selectively probed experimentally using a two-colour femtosecond pump–probe photoionisation scheme. Both photo-ion and photo-electron signals have revealed a dissociation dynamics proceeding on two distinct time scales ($\sim$30~ps and 700~ps). By comparing with time-dependent DFT simulations, we concluded that the fast dynamics was due to prompt desorption of Rb atoms when exciting the $^2\Sigma_{1/2}$-state. The formation time of the surface bound excimer observed in these simulations upon excitation to the $^2\Pi_{3/2}$ state was between 20~ps and 50~ps. By introducing a $^2\Pi_{3/2}\!\!\longrightarrow\!\!^2\Pi_{1/2}$ spin-orbit relaxation process in the time-dependent DFT simulations we were able to reproduce dissociation of this exciplex from the droplet surface. Fitting the experimental results gave a lifetime of $\sim$700~ps for spin-orbit relaxation.

The second part of the results was devoted to helium droplet doping, and the influence of the existence of vortex lines on this process. In Chapter~\ref{ch:head-on-xece} we have shown that head-on collisions of helium nanodroplets with xenon, a heliophilic atom, involve a kinetic energy exchange of the same order of magnitude as cesium, a heliophobic atom with similar mass. In both cases, this energy is largely dissipated by producing energetic waves in the droplet or it is carried away by promptly emitted helium atoms. The difference between the two atoms is due to the different nature of their interaction with helium. Density build up is observed around the heliophilic xenon during the dynamics, whereas a bubble is created around the heliophobic cesium. Thus it takes a much higher velocity for xenon to go through the droplet and escape than for cesium, as could be expected.

And finally in Chapter~\ref{ch:capture} we have demonstrated that Xe and Ar atoms at thermal velocities are readily captured by helium droplets, with a capture cross section similar to the geometric cross section of the droplet. Crucially for the subsequent capture of impurities by vortex lines, we have also shown that most of the kinetic energy of the impinging impurity is lost in the capture process during the first tens of picoseconds. This happens either by the ejection of prompt-emitted He atoms, or by the production of sound waves and large deformations in the droplet. 

In addition, we have also shown that if the droplet hosts a vortex, slowly moving impurities are readily captured by the vortex line. Rather than being trapped inside the vortex core, the impurity is bound to move at a close distance around it. Besides the crucial energy loss when the impurity hits the droplet, the capture by the vortex is favoured by a further energy transfer from the impurity to the droplet: large amplitude displacements of the vortex line ---as shown in the ESI\citep{ESI}--- take place, constituting another source of the kinetic energy loss in the final stages of the capture. A related issue is the appearance of Kelvin modes in the vortex line, that is not only bent, but also twisted in the course of the collision.

We can conclude that if the kinematic conditions of the collision (kinetic energy and impact parameter) lead to the capture of the impurity by the droplet, the pinball effect caused by the droplet surface can facilitate the meeting of the Xe/Ar atom and the vortex line ---and the possible capture of the atom by the vortex--- since both have a tendency to remain in the inner region of the droplet. We have shown this in the case of Xe at $v_0$=200~m/s: Xe is captured during its second transit across the droplet, whereas this could not have happened in bulk liquid helium\citep{Psh16}. This effect could explain the capture of impurities by vortex lines even in the very large droplets used in the observation of filament-shaped nanostructures.

\section*{Future prospects}
Our work on the real-time dynamics of the photo-excitation of an alkali metal atom on the surface of a helium nanodroplet was quite extensive. Conducting the same type of studies on other types of dopant species which are solvated more deeply inside He droplets (e.g. alkaline earth metals, transition metals) would give further insight into the mechanisms of desolvation and ejection of excited impurity atoms out of He nanodroplets\citep{Loginov:2007,Loginov:2012, Kautsch:2013,Lindebner:2014}.

Moreover, a more complete description of the couplings between electronic states and the configurational degrees of freedom in such excited complexes induced by the He droplet environment would be highly desirable\citep{Closser:2014,Masson:2014}. In a recent advance, electronic relaxation of Ba$^+$ cations in He nanodroplets, based on a diabatisation of the He–Ba$^+$ ground and excited electronic states interaction potentials\citep{Vindel:2018}, has been proposed as a mechanism for ejecting Ba$^+$ and Ba$^+$He$_n$ off He droplets. These mechanisms for spin-relaxation and inter-electronic state relaxation have to be confirmed by real-time dynamics studies.

Finally, the capabilities of the He-DFT approach might help elucidate processes of experimental interest, such as the capture of one or several impurities by large droplets hosting a vortex array and how several atomic impurities, impinging upon a rotating droplet hosting vortices, react to form small clusters, eventually being trapped within the vortex cores as shown by the appearance of filament-shaped nanostructures in experiments.

In all these future lines of investigation, $^4$He-DFT and TDDFT will be essential tools, given their ability to accurately describe the equilibrium and dynamics properties of realistic size helium nanodroplets, possibly hosting vortices, in interaction with dopants. In this context it will be extremely interesting to couple $^4$He-TDDFT with quantum molecular dynamics in order to go beyond the mean field approximation of the dopant dynamics in the helium environment. A promising alternative in the correlated basis function (CBF) approach and its multi-component approach recently developed by Rader \textit{et al.}\citep{Rader2017}