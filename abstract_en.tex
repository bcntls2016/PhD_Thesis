\chapter{Abstract}
	In this thesis we investigate two aspects of the dynamics of atomic impurities interacting with superfluid helium (He) nanodroplets, namely the photo-excitation of alkalis on a nanodroplet and the doping process of nanodroplets hosting quantised vortices with noble gas atoms. For the theoretical investigations we use He density functional theory and its time-dependent version.

	The first aspect involves a joint experimental and theoretical collaboration that focusses on the photo-excitation of the alkali rubidium (Rb). Alkalis are a very interesting probe of He droplets since they reside in their surface region, where it has been argued that almost 100\% Bose-Einstein condensation could be achieved due to a density that is lower than in bulk superfluid He.

	In our simulations we find that states excited to the 5p and 6p manifold desorb at very different timescales, separated by 2 orders of magnitude ($\sim$100 ps and $\sim$1 ps for 5p and 6p respectively). This is in good agreement with experimental results where the desorption behaviour of photo-excited Rb atoms is determined using a femtosecond pump-probe scheme. However, in our simulations excitation to the $5\mathrm{p}\,^2\Pi_{3/2}$-state results in a surface-bound RbHe exciplex, contrary to the experimental case where the RbHe exciplex desorbs from the droplets surface. Introducing $^2\Pi_{1/2}\leftarrow{^2}\Pi_{3/2}$ spin-relaxation into the simulations, the RbHe exciplex is able to desorb from the droplet's surface, which resolves this contradiction.

	The second aspect concerns a purely theoretical investigation that is inspired by recent work of Gomez and Vilesov \emph{et al}., where quantised vortices were visualised by doping He nanodroplets with silver atoms, subsequently ``soft landing'' them on a carbon screen. Electron-microscope images show long filaments of silver atom clusters that accumulated along the vortex cores. Also the formation of quantum-vortex lattices inside nanodroplets is evidenced by using X-ray diffractive imaging to visualise the characteristic Bragg patterns from xenon (Xe) clusters trapped inside the vortex cores.

	First, head-on collisions between heliophilic Xe and a He nanodroplet made of 1000 He atoms are studied. The results are then compared with the results of a previous study of an equivalent kinematic case with cesium (Cs), which is heliophobic. Xe acquires a ``snowball'' of He around itself when it traverses the droplet and much more kinetic energy is required before Xe is able to pierce the droplet completely. When it does, it takes away some He with it, contrary to the Cs case.

	Next, collisions between argon (Ar)/Xe and pristine superfluid He nanodroplets are performed for various initial velocities and impact parameters, to determine the effective cross-section for capture. Finally, the simulations are then repeated for droplets hosting a single quantised vortex line. It is observed that the impact of the impurities induces large bending and twisting excitations of the vortex line, including the generation of helical Kelvin waves propagating along the vortex core. We conclude that Ar/Xe is captured by the quantised vortex line, although not in its core. Also we find that a He droplet, hosting a 6-vortex line array whose cores are filled with Ar atoms, results in added rigidity to the system which stabilises the droplets at low angular velocities.

	Our simulations involving droplets hosting quantum vortices open the way to further investigations on droplets hosting an array of vortices, involving multiple impurities.
