\chapter{Angular velocity and angular momentum}

\lettrine[lines=3]{\color{activeColor}I}{n} this Appendix we discuss the relationship between angular velocity and angular momentum of a deformed droplet below the critical angular frequency for vortex 
nucleation.

Let us consider an ellipsoidal vessel filled with liquid $^4$He uniformly rotating around the $z$ axis, $\omega = \omega\, \hat{\mathbf{k}}$.
The ellipsoid has the equation
 %
$$ F(x,y,z) =\frac{x^2}{R_1^2}+ \frac{y^2}{R_2^2}+  \frac{z^2}{R_3^2} - 1 = 0$$
%
If $\mathbf{v}$ is the  irrotational velocity of a point in the laboratory, $\mathbf{v}'$ the velocity of the same point in the vessel (corotating frame),  and 
$\mathbf{V}= \omega \times \mathbf{r}$, one has
%
$$\mathbf{v}'=  \mathbf{v} - \mathbf{V} = \frac{\hbar}{m_4} \, \nabla {\cal S} - \omega \times \mathbf{r}$$
%
where ${\cal S}$ is the velocity potential defined here so as  that 
%
$$\mathbf{v}= \frac{\hbar}{m_4} \, \nabla {\cal S}(x,y,z)$$
%
Its existence is granted by irrotationality;  we also have
  $\mathbf{V}= \omega \times \mathbf{r} = \omega (-y,\, x, \,0)$.
A vector  perpendicular to the ellipsoid  surface is 
%
$\mathbf{n} = \nabla  F(x,y,z)$.
%
From the stationarity condition 
%
$( \mathbf{v}'  \cdot  \mathbf{n})|_{surf} =0$
%
one obtains
%
\begin{eqnarray}
&&\mathbf{v}'  \cdot  \mathbf{n}= 0 = \left.\left(\frac{\hbar}{m_4} \, \frac{\partial {\cal S}}{\partial x} + \omega y\right) \frac{x}{R_1^2} \right.
\nonumber
\\
\nonumber
&&
+  \left. \left(\frac{\hbar}{m_4} \,\frac{\partial {\cal S}}{\partial y} - \omega x\right) \frac{y}{R_2^2} 
+  \left(\frac{\hbar}{m_4} \, \frac{\partial {\cal S}}{\partial z} \right) \frac{z}{R_3^2} \right|_{surf}
\end{eqnarray}
%
It can be checked that ${\cal S} = \alpha xy$ is a solution to this equation provided that
%
$$\frac{\hbar}{m_4} \,  \left( \frac{1}{R_1^2}  + \frac{1}{R_2^2} \right)  \alpha = \left(\frac{1}{R_2^2}  - \frac{1}{R_1^2} \right)  \omega$$
%
Hence,
%
$$ \alpha = \frac{m_4}{\hbar}\,\left( \frac{R_2^2-R_1^2}{R_1^2+R_2^2}\right) \, \omega$$
%
and
%
$$ {\cal S} = \frac{m_4}{\hbar} \, \left( \frac{R_2^2-R_1^2}{R_1^2+R_2^2}\right)  \, \omega \, xy$$
%
The velocity in the laboratory is 
$ \mathbf{v} = (\hbar/m_4)\nabla {\cal S}=  (\hbar/m_4)\alpha (y, x, 0) $,
and in the vessel (corotating frame) is
$\mathbf{v}'= \beta (R_1^2 y, \, -R_2 ^2 x, \,0) $, where $\beta \equiv  2 \, \omega/(R_1^2+R_2^2)$.
 Once they have been determined, their  circulation lines are straightforwardly obtained. 
In the laboratory frame they are
$$ x^2 - y^2 = c \quad ,$$
%
which is the appearance of the circulation lines displayed in \fig{fig8-capture}. In the vessel  frame, they are
%
$$ \frac{x^2}{(\xi R_1)^2}+ \frac{ y^2}{(\xi R_2)^2} = 1 \quad .$$ 
%
These lines are ``parallel'' to the ellipsoidal surface.

We define the deformation parameter $\epsilon$
%
$$
\epsilon =\frac{\langle x^2 \rangle - \langle y^2 \rangle}{\langle x^2 \rangle + \langle y^2 \rangle}
$$
%
where e.g., 
$$ \langle x^2 \rangle = \frac{1}{N} \int d\mathbf{r} \,  x^2 \,\Psi(\mathbf{r}) $$
%
For the sharp surface ellipsoid above,
%
\begin{equation}
 \alpha = \frac{m_4}{\hbar} \, \epsilon \, \omega
 \label{eps}
 \end{equation}
%
This relationship is not  general but  can be used as a guide for our more general approach.
Let us  now discuss the angular momentum and moment of inertia of the irrotational fluid droplet. Recalling that
%
$$L_z = - \imath \; \hbar \left( x \frac{\partial}{\partial y} -  y \frac{\partial}{\partial x}\right)$$
%
if we write 
%
$$\Psi(\mathbf{r}) = \Phi(\mathbf{r}) e^{\imath \alpha xy}$$
%
with $\Phi(\mathbf{r})$ a real function,
%
$$\langle L_z \rangle = \hbar\,  \alpha  \int d\mathbf{r} \,  (x^2-y^2)  \,\Phi^2(\mathbf{r})$$
%
If \eq{eps} holds,
%
\begin{eqnarray}
&&\langle L_z \rangle = \epsilon \, m_4 N  [\langle x^2 \rangle - \langle y^2\rangle]\, \omega   
\nonumber
\\ 
&& = m_4  N \left( \frac{[\langle x^2 \rangle - \langle y^2\rangle ]^2}{\langle x^2 \rangle + \langle y^2\rangle} \right)\, \omega  \equiv  {\cal I}_{irr} \, \omega
 \label{irrot}
\end{eqnarray}
%
where 
%
$$ {\cal I}_{irr} = m_4 N \left( \frac{[\langle x^2 \rangle - \langle y^2\rangle ]^2}{\langle x^2 \rangle + \langle y^2\rangle} \right)$$
%
is the irrotational moment of inertia. For a rigid solid,
%
$$  {\cal I}_{rig} =  m_4  \int d\mathbf{r} \,  (x^2+y^2)  \,\Phi^2(\mathbf{r}) = m_4 N [\langle x^2 \rangle + \langle y^2\rangle] $$
%
Hence, 
%
$$ \frac{{\cal I}_{irr}}{{\cal I}_{rig}} = \left[ \frac{\langle x^2 \rangle - \langle y^2\rangle }{\langle x^2 \rangle + \langle y^2\rangle} \right]^2 \rightarrow 0 \quad {\rm if} \quad \epsilon \rightarrow 0$$
%
Finally, we discuss the kinetic energy of the droplet
%
$$ E_{kin} = \frac{\hbar^2}{2 m_4}  \int \, d\mathbf{r} \,|\nabla \Psi(\mathbf{r})|^2$$
%
From the above $\Psi(\mathbf{r})$, 
%
\begin{eqnarray}
&&E_{kin}= \frac{\hbar^2}{2 m_4}  \int \, d\mathbf{r} \,|\nabla \Phi(\mathbf{r})|^2
\nonumber
\\
\nonumber
&+& \frac{\hbar^2}{2 m_4} \alpha^2 \int \, d\mathbf{r} \, (x^2 + y^2) \Phi^2(\mathbf{r})= E_{intr}+E_{coll}
\end{eqnarray}
%
where the first term is the ``intrinsic'' kinetic energy and the second term arises from the irrotational velocity field
%
\begin{eqnarray}
&&E_{coll}= \frac{\hbar^2}{2 m_4} \alpha^2 \int  d\mathbf{r} \, (x^2 + y^2) \Phi^2(\mathbf{r}) =
\nonumber
\\
\nonumber
&& \frac{1}{2} \left\{ m_4\, \epsilon^2 \int d\mathbf{r} \, (x^2 + y^2) \Phi^2(\mathbf{r})\right\} \omega^2= \frac{1}{2}\, {\cal I}_{irr} \,\omega^2
 \end{eqnarray}
% 

 These expressions may be used to obtain some estimates from the actual DFT calculations.
 For  a $^4$He$_{1000}$ droplet and $\omega$=0.10 K/$\hbar$ 
 we have obtained $\langle x^2\rangle$= 100.82 \AA$^2$ and  $\langle y^2\rangle$= 101.82 \AA$^2$; hence,
 $\epsilon \sim -1/200$. Since $\hbar^2/m_4$= 12.12 K \AA$^2$, from \eq{eps} one has $\alpha \sim -4.2 \times 10^{-5}$ \AA$^{-2}$. 
From \eq{irrot} we obtain  $\langle L_z\rangle \sim 4 \times 10^{-2} \, \hbar$.
  
 In a Bose-Einstein condensate  the deformation $\epsilon$ is a control parameter that can be set to a very large value (close to unity). For a self-bound $^4$He droplet deformation comes from
 ``rotation'' itself and it turns out to be minute
  even for angular frequencies close to the critical frequency  for one-vortex nucleation; the conclusion is that the droplet ``does not rotate'';
 in other words, it is unable to store an appreciable amount of angular momentum before vortex  nucleation.
