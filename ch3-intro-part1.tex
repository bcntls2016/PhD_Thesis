\chapter{Introduction}
	\section{Imaging Excited-State Dynamics}
		\lettrine[lines=3,findent=3pt,nindent=0pt]{A}{s} mentioned in the introduction He nanodroplets capable of efficiently capturing and cooling atoms, molecules, and clusters for spectroscopy and dynamics studies[1,2]. So far, time-resolved experiments on He droplets doped with alkali (Ak) metal atoms were mostly focused on the formation of AkHe exciplexes induced by laser excitation[15-18,22,23]. The concurrent desorption of these excited species was estimated to proceed on a picosecond time scale[9,10,13,24-26]. This estimate, sufficient for studies employing nanosecond laser pulses, clearly lacks precision for experiments with sub-picosecond time resolution. Thus in the previous measurements as well as in experiments focusing on electronic and vibrational coherences of Ak atoms and molecules[17,19,22,23,27] the exact location of the dopants, attached to the droplets or in the vacuum, has remained somewhat uncertain.\\

		Here we report a combined experimental and theoretical investigation of the excited-state dynamics of doped He nanodroplets in real time. The combination of fs pump-probe spectroscopy with velocity map imaging (VMI)[28] allows us to clearly disentangle complex formation, desorption, and ion solvation. As a model system, we investigate He droplets doped with single rubidium (Rb) atoms. Ground-state Rb atoms and small molecules are weakly bound to the He droplet surface in a dimple structure[29-31]. Therefore, the ejection dynamics of the excited Rb atom (Rb*) is not affected by processes such as the interaction of Rb* with density waves traveling in the bulk of the droplet, as for Ag*[11].

	\section{Desorption dynamics of RbHe-exciplexes}
		\lettrine[lines=3,findent=3pt,nindent=0pt]{U}{nderstanding} the photochemistry of condensed phase systems and surfaces is essential in many research areas, such as atmospheric sciences[1] and photocatalysis[2]. However, complex diabatic couplings of electronic and motional degrees of freedom of various subunits of the system often present a major challenge. Moreover, the heterogeneity of multi-component solid or liquid systems and experimental difficulties in precisely preparing the sample and reproducing measurements tend to make it hard to unravel specific elementary reactions. In this respect, He nanodroplets doped with single atoms or well-defined complexes are ideal model systems for studying photodynamical processes in the condensed phase, both experimentally and theoretically. Due to their ultra-low temperature (0.37 K) and their quantum fluid nature, He nanodroplets have a homogeneous density distribution and dopant particles aggregate into cold clusters mostly inside the droplets[3,4]. Only alkali metal atoms and small clusters are attached to He droplets in losely bound dimple-like states at the droplet surface[5-12].\\

		While He nanodroplets are extremely inert and weakly-perturbing matrices for spectroscopy of embedded atoms and molecules in their electronic ground state, a rich photochemical dynamics is initiated upon electronic excitation or ionisation[13,14], involving electronic relaxation[15-18], the ejection of the dopant out of the droplet[19-26], chemical reactions within the dopant complex[27-29], and even among the dopant and the surrounding He[20,30-39].\\

		As a general trend, electronically excited dopant atoms and small molecules tend to be ejected out of He droplets either as bare particles or with a few He atoms attached to them[15,40-42]. In particular, all atomic alkali species promptly desorb off the droplet surface, the only exceptions being Rb and Cs atoms in their lowest excited states[43,44]. The dynamics of the desorption process has recently been studied at an increasing level of detail[16,21,23,45], including time-resolved experiments and simulations[24,26]. The focus was on the competing processes of desorption of the dopant induced by laser excitation, and the dopant falling back into the He droplet upon photoionisation. The latter occurred at short pump-probe delay times when the distance between the photoion and the droplet was short enough for ion-He attraction to be effective.\\

		The purpose of this work is to extend our joint experimental and theoretical study of the photodynamics of Rb-doped He nanodroplets to RbHe exciplexes[23,26]. The simultaneous effect of pairwise Rb-He attraction and repulsion of Rb from the He droplet as a whole results in an intricate dynamics, and interpretations have remained somewhat ambiguous with respect to the exciplex formation mechanism and time scale, as well as the origin of free exciplexes detached from the He droplets[16,26,30,33,34]. In particular the role of relaxation of internal degrees of freedom of the RbHe exciplex in the desorption process has not been explicitly addressed[46,47]. Here, we discuss in detail the interplay of the RbHe formation dynamics, the RbHe desorption off the He droplet surface, and the fall-back of [RbHe]+ created by photoionization in femtosecond pump-probe experiments[24,26,31]. We find that electronic spin-relaxation is the main process driving the desorption of RbHe off the He droplet.
		
	\section{Supervised work: potassium-doped\\nanodroplets}
		\lettrine[lines=3,findent=3pt,nindent=0pt]{U}{nder} Nadine Halberstadt's and my supervision, a master stage -- \emph{M2 Physique Fondamentale} -- titled ``\emph{\textbf{Dynamics of a superfluid helium nanodroplet doped with a single potassium atom}}'' has been performed by Maxime Martinez.\\
		
		The project investigates the static and dynamic behaviour of a single potassium atom excited from the K-$^4$He$_{1000}$ equilibrium configuration to the K*(4p)-$^4$He$_{1000}$ and K*(5s)-$^4$He$_{1000}$ states. The choice of potassium was motivated by a discrepancy in the time-resolved experimental studies\citep{Schulz2001,Reho2000-1,Reho2000-2}. Moreover, the mass of potassium sits between those of the heavier alkalis like rubidium and cesium, and the lighter ones, like lithium and sodium. Therefore, potassium presents an interesting case, being on the borderline between the classical regime for heavy alkalies and a quantum--mechanical regime for the lighter ones. Both treatments of the equilibrium properties and the 5s$\leftarrow$4s excitation are studied. This work is not included in the thesis but can be found here \citep{Martinez2017}.