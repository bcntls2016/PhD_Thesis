%\chapter{Summary}\markboth{\sectionfont\itshape\color{activeColor}Summary}{} %%% IN FRONTMATTER
%\chapter{Summary} %%% IN MAINMATTER
\chapter{Summary}\fancyhead[RE,LO]{\subsectionfont\itshape\color{activeColor}Summary} %%% IN BACKMATTER
	\lettrine[lines=4]{\color{activeColor}I}{n} this thesis we investigate two aspects of the dynamics of atomic impurities interacting with superfluid helium (He) nanodroplets, namely the photo-excitation of alkalis on a nanodroplet and the doping process of nanodroplets hosting quantised vortices with noble gas atoms. For the theoretical investigations we use He density functional theory (He-DFT) and its time-dependent version (He-TDDFT).

	The first aspect involves a joint experimental and theoretical collaboration that focusses on the photo-excitation of the alkali rubidium (Rb). Alkalis are a very interesting probe of He droplets since they reside in their surface region, where it has been argued that almost 100\% Bose-Einstein condensation could be achieved due to a density that is lower than in bulk superfluid He.

	The second aspect concerns a purely theoretical investigation that is inspired by recent work of Gomez and Vilesov \emph{et al}., where quantised vortices were visualised by doping He nanodroplets with silver atoms, subsequently ``soft landing'' them on a carbon screen. Electron-microscope images show long filaments of silver atom clusters that accumulated along the vortex cores. Also the formation of quantum-vortex lattices inside nanodroplets is evidenced by using X-ray diffractive imaging to visualise the characteristic Bragg patterns from xenon (Xe) clusters trapped inside the vortex cores.

	Our simulations involving droplets hosting quantum vortices open the way to further investigations on droplets hosting an array of vortices, involving multiple impurities.
		
	\subsection*{Introduction (section 1.3 in the thesis)}
		Until the 1980's, most experimental and theoretical work was done on bulk systems, i.e. systems of the order of $N_A$ number of atoms. It was only in the last couple of decades that advancements in technology enabled experimentalists to create nanoscale sized superfluid helium droplets. From the early 1990's onwards, superfluid helium nano-droplets became an active field of study, both experimentally and theoretically. 
		
		Helium nanodroplets are considered ideal model systems to explore quantum hydrodynamics in self-contained, isolated superfluids. The main focus has been on the evolution of their properties with the number of atoms in the cluster, until the condensed matter limit is reached. Helium clusters are especially interesting in that quantum effects play a key role in determining their properties. In particular, given that a helium cluster is an ensemble of bosons at about 0.4 K\citep{Brink1990,Hartmann1995}, manifestations of collective behaviour (such as superfluidity) are expected. On the other hand, it is not yet clear how the finite size of a cluster affects this non-classical (or degenerate) collective behaviour.

		Recently, Toennies et al.\citep{Hartmann1996} have measured the electronic spectrum of glyoxal molecules embedded in He clusters and found it consistent with a theoretical simulation computed using the phonon dispersion curve of superfluid bulk He II. The authors themselves, however, point out that at the average cluster size of 5500 He atoms reported in\rf{Hartmann1996}, the clusters are so big that finite size effects in the interior region are negligible (see also\rfs{RamaKrishna1990-1,Chin1992}). It is therefore not surprising that they find results consistent with the bulk case, especially for a molecule readily solvated inside the cluster, for which surface effects play a minor role. Therefore the influence of the He clusters size on superfluidity has not been detected so far.
		
		The helium-helium interaction is already weak in bulk liquid helium and in finite self-bound systems such as droplets it is even weaker, e.g. the binding energy per atom is $<$7.17~K. Because of this helium droplets cool down very rapidly due to fast evaporation and therefore reaching their limiting temperature of about $0.38\unit{K}$ in microseconds. Pure helium droplets are neutral systems and their properties like their size, binding energy and excitation spectra, are not easy to determine experimentally and are usually obtained by indirect methods. This did not stop the theoreticians describe doped $^4$He$_N$ droplets using a wide variety of approaches depending on the size and character of the droplets ranging from Quantum Monte Carlo, Hypernetted-Chain/Euler-Lagrange\citep{Krotscheck2001}, Variational Monte Carlo\citep{Gartner2018} and many others.
		
		A key property of helium droplets, in contrast to bulk helium, is their ability to pickup any kind of dopants with which they collide. Depending on the strength of the dopant-$^4$He interaction and the surface tension of the droplet, a dimensionless parameter $\lambda$ can be defined\citep{Anc95} with a critical value $\lambda_0\sim1.9$. Below $\lambda_0$ impurities are bound to the surface of the droplet (e.g. the alkalis), and above they get solvated into the droplet's interior. Droplets can therefore be doped with almost any kind of atomic- or molecular species. 
		
		From the perspective of the droplet it this means that it is possible to use the dopants as gentle probes to determine the superfluid properties of helium droplets that would be inaccessible with other methods. For two examples of this see\rfs{Gre98,Sin89,RamaKrishna1990-2}, where a dopant is used to probe the superfluid character of small $^4$He droplets and\rfs{Hartmann1999,Harms1999} to see their limiting temperatures.
		
		Moreover, from the perspective of the impurities it enables a broad spectrum of possible experimental studies. Due to the fact that helium droplets are ultra cold superfluid liquids, and therefore provide high mobility of any picked-up dopants, one can conduct high resolution spectroscopy studies. Having a fine control over the number of picked-up dopants[29] one can use droplets as a matrix for creating self-organising structures of polar molecules, or very cold metal clusters and study their Coulomb explosion.
				
		One of the most intriguing properties of superfluid helium droplets is the fact that they can host quantised vortices. Because of their ultra low temperature they are true quantum liquids and their vorticity and angular momentum are quantised. The existence of quantised vortices was anticipated because they have been created and observed in BECs made of dilute gases. However, the detection of quantised vortices is still experimentally challenging (see \scn{sec:quant-vort} in this thesis).
		
		A lot of of work has been done on helium droplets the last few decades, both experimentally and theoretically. From the absorption spectra of alkali metal doped helium droplets, the study of doped mixed $^3$He--$^4$He droplets, electrons in liquid helium, to the investigation of the critical Landau velocity inside small $^4$He droplet. For a comprehensive overview of work done in the last two decades, the interested reader is referred to the review papers in\rfs{Barranco2006,Ancilotto2017,Mudrich2014}.
	
	\subsection*{Methods used (sections. 2.1, 2.2)}
		From a theoretical point of view, superfluid helium must be considered as a high dimensional quantum system. Quantum Monte Carlo\citep{Kro02} (QMC) and direct quantum mechanical \citep{deL06,deL10,Agu13} calculations are the most accurate methods, but their computational demand quickly exceeds currently available computer resources when the number of helium atoms increases. Furthermore, QMC cannot describe dynamic evolution of superfluid helium in real time. To address these limitations, semi-empirical methods based on the density functional theory (DFT) formalism have been introduced\citep{Str87a,Str87b,Dal95}. DFT can be applied to much larger systems than QMC and allows for time-dependent formulation. As such, it offers a good compromise between accuracy and computational feasibility. The main drawback of DFT is that the exact energy functional is not known and must therefore be constructed in a semi-empirical manner. Moreover, doped helium droplets are limited to a mean-field description of the dopant-helium interaction. Nevertheless, DFT is the only method to date that can successfully reproduce results from a wide range of time-resolved experiments in superfluid helium, for realistic sizes compared to experimental conditions.
		
		The starting point for the density functional method is the Hohenberg-Kohn (HK) theorem\citep{Hohenberg1964}, which states that the ground-state energy $E_v$ of an \emph{interacting inhomogeneous} system in a static potential $v$ can be written as a unique functional of the one-body density $\rho$ as $F[\rho]$; a universal functional ---valid for \emph{any} number of particles and \emph{any} external potential $v$--- of the density.
		
		Kohn and Sham (KS) later reformulated\citep{Kohn1965} the theory by introducing an approximation scheme for the functional $F[\rho]$ that is analogous to Hartree's method, but also contains the major part of the correlation effects inherent in interacting many-body systems. The approximation starts by splitting the functional in a kinetic energy- and a correlation energy part. The kinetic energy is that of a fictitious system of \emph{non-interacting} particles with density $\rho$. The correlation part corresponds to an \emph{interacting} system with the same density. For the kinetic part this allows us to write the total kinetic energy as the sum of the individual kinetic energies of the non-interacting particles. There is difference between the true kinetic energy of the interacting system and the fictitious one, due to the neglecting of the correlations. This difference is corrected and accounted for in the correlation energy part. It is only the sum of the two that gives the correct ground state energy of the interacting system of particles.
		
		Because the functional we used in this work is calibrated to produce the correct behaviour of bulk liquid helium at zero temperature and zero pressure, we assume complete Bose-Einstein (BE) condensation of the helium. In this case all the helium atoms occupy the same single-particle ground state KS-orbital. Therefore the many-body wave function and the density further simplify and allow us to describe the whole condensate by defining an effective wave function that only depends on one coordinate in a 3-dimensional Cartesian space. 
		
		The difficult job is to design a functional such that the desired physical properties of helium can be recovered. This is far from trivial but several of these density functionals are available now. The density functional used in this work is based on the \emph{Orsay-Trento} density functional which is discussed in \scn{sec:otdft} of the thesis. It uses a finite-range, non-local approach and it is, to date the most accurate model in the sense that its parameters were fitted to reproduce the bulk properties of liquid helium at zero temperature.
		
		In the presence of highly inhomogeneous liquid densities, e.g. atomic impurities with a very strong He-X interaction, the OT-functional becomes numerically unstable. To deal with this problem an additional energy penalty term is imposed. Including this term in the OT-functional prevents excessive density build-up. Dropping the non-local terms of the original OT-functional and adding the penalty term gives a modified density functional that is referred to as the \emph{Solid Functional}. See \ref{sec:solid} for more details and its parameters.
		
		To describe the time evolution of the system, the Runge-Gross theorem extends DFT to its time-dependent version TDDFT\citep{Run84}. The functional variation of an associated action (see \eq{eq:action in the thesis} for an example) leads to a time-dependent Euler-Lagrange (EL) equation. Considering only stationary states of the Hamiltonian a time-independent EL-equations is obtained that, upon solving, gives the ground-state energy of the system.
		
		For more details on how the static- and dynamic calculations are solved for the various impurities in isotropic- and anisotropic interaction potentials, please confer to Sections~2.3~and~2.4.
		
	\subsection*{Excited state dynamics of alkali-doped nanodroplets}
		In a 1996 paper\citep{Griffin1996} Griffin and Stringari have argued that almost 100\% Bose-Einstein Condensation could be achieved in the low density surface region of superfluid He at $T=0$, as opposed to only about 10\% in the bulk. It is therefore evident that a minimally perturbing probe capable of investigating the surface of a He cluster is very desirable.

		It was argued from a theoretical perspective\citep{Dalfovo1994} that the alkali atoms reside on the cluster surface. Experimental evidence for this was found\citep{Stienkemeier1995-1,Stienkemeier1995-2,Ancilotto1995-1} later when it was observed that the laser induced fluorescence (LIF) spectrum of sodium was shifted compared to sodium in the gas phase due to the presence of the He cluster. However, not as much as alkali atoms in the bulk of liquid helium.
	
		It comes as no surprise then that alkali atoms are a very natural choice for exactly these type of studies.  For example, with a solvation parameter (see \scn{sec:helium-droplets}) of $\lambda=0.729$\citep{Anc95}, Rb will remain bound to the surface of the droplet. Furthermore, alkalis have a simple, well known, absorption spectrum. Moreover, their simple, one-valence electron structure allows for detailed theoretical modelling. They introduce only weak perturbations (alkali-helium interaction energies are on the order of 1 cm$^{-1}$\citep{Pat91}). Lastly, theoretical calculations\citep{Ancilotto1995-2,Kanorsky1994} and experimental spectra\citep{Tabbert1995,Takahashi1993,Beijersbergen1993} of alkali atoms in bulk liquid helium are available for comparison.
	
		Given that alkalis are ideal objects to probe the boundary region of the nanodroplets, the $n\mathrm{p}\,^2\mathrm{P}\!\longleftarrow\!n\mathrm{s}\,^2\mathrm{S}$ transitions of the alkali atoms have attracted much interest from an experimental as well as a theoretical point of view. The spectroscopy of higher excited states has been thoroughly explored\citep{Log11b,Log11a,Lackner2012,Lackner2013,The11,Fec12,Pif10,Lac11,Theisen2011,Lac13}. The obtained spectra can be successfully reproduced by a pseudo-diatomic model, except for the higher excited states, where the model progressively fails due to the limitations imposed by its realm of validity\citep{Sti96,Bunermann2007}. While the the effect of the excited states on the spectra are now fairly well understood, their influence on the following dynamics is largely unexplored.
	
		In this part of the thesis, the results of the real-time dynamics of a single electronically excited rubidium (Rb) atom residing in the surface dimple of a helium nano-droplet are presented. The atom is excited from its ground state 5s$\,^2\Sigma_{1/2}$ to the 5p$\,^2\{\Sigma,\Pi\}$ and 6p$\,^2\{\Sigma,\Pi\}$ manifold (see \scn{sec:dim-model} for an explanation of the used electronic state labels). Usually they desorb upon excitation either as a bare atom or as a complex with one or more helium atoms, called an ``exciplex''.	
	
	\subsection*{Imaging excited-state dynamics}	
		The following article is a combined experimental and theoretical study focussing on imaging and characterising the dynamics following the 5p$\leftarrow$5s and 6p$\leftarrow$5s excitations of rubidium hosted by a helium nanodroplet. The experiment used femtosecond pump-probe techniques with a first laser exciting the Rb on the droplet surface at time $t_{exc}$ and a second laser ionising it for detection with VMI at time $t_{ion}$. The results characterised a critical time delay, called the ``fall-back time'', between two opposite outcomes. If $t_{ion}-t_{exc}\leq\tau$, the departing Rb atom is still rather close to the droplet when the probe laser turns their interaction to attractive. As a result, the Rb$^+$ turns around and gets solvated. On the other hand, for $t_{ion}-t_{exc}\geq\tau$, ionisation occurs too late for Rb$^+$ to feel an appreciable attraction from the droplet, and it had already too much kinetic energy, so that it escapes. 
	
		The theoretical study focussed on understanding the desorption dynamics and determining the fall-back times to compare with the experiment. It made use of the He-TDDFT presented in \scn{sec:td-dft}, both in the excited and ionised states. The results are presented in the following article which was published in the Journal of Physical Chemistry Letters\citep{Vangerow2017}.
		
		In our simulations we find that states excited to the 5p and 6p manifold desorb at very different timescales, separated by 2 orders of magnitude ($\sim$100 ps and $\sim$1 ps for 5p and 6p respectively). This is in good agreement with experimental results where the desorption behaviour of photo-excited Rb atoms is determined using a femtosecond pump-probe scheme. 

	\subsection*{Desorption dynaics of RbHe exciplexes}
		However, in our simulations excitation to the $5\mathrm{p}\,^2\Pi_{3/2}$-state results in a surface-bound RbHe exciplex, contrary to the experimental case where the RbHe exciplex desorbs from the droplets surface. Introducing $^2\Pi_{1/2}\leftarrow{^2}\Pi_{3/2}$ spin-relaxation into the simulations, the RbHe exciplex is able to desorb from the droplet's surface, which resolves this contradiction.
		
		How do we resolve the discrepancy between the experimental observation that Rb atoms, excited to the 5p$\,^2\Pi_{3/2}$ state, detach from the droplet surface, and TD-DFT simulations that show that they result in a surface-bound state? That is the question that led to this work. Upon photo-excitation of Rb to the 5p$\,^2\Pi_{3/2}$ state, a He atom may be attached to it forming a HeRb exciplex; this cannot happen if Rb is excited to the 5p$\,^2\Pi_{1/2}$ state because it finds a barrier (see \fig{fig:potentials}) preventing exciplex formation.

		In the gas phase, a HeRb 5p$\,^2\Pi_{1/2}$ exciplex can be formed if there is enough kinetic energy for Rb* to overcome the potential barrier; alternatively, the collision of the HeRb 5p$\,^2\Pi_{3/2}$ exciplex with another atom or complex might relax the Rb* atom from the 5p$\,^2\Pi_{3/2}$ to the 5p$\,^2\Pi_{1/2}$ state, overcoming the barrier as the potential wells for both states are at similar Rb-He distances. In the condensed (droplet) phase at 0.4 K temperature, neither of these mechanisms are available to explain the formation of HeRb 5p$\,^2\Pi_{1/2}$ exciplexes and their potential ejection.

		However, another possible way for this to happen is non-radiative de-excitation from the 5p$\,^2\Pi_{3/2}$ to the 5p$\,^2\Pi_{1/2}$ that populates the latter state and leaves the Rb* atom with enough kinetic energy so as to be ejected. Notice from \fig{fig:potentials} that the minimum of the 5p$\,^2\Pi_{3/2}$ potential is 12683 cm$^{-1}$, and that of the 5p$\,^2\Pi_{1/2}$ potential is at 12518 cm$^{-1}$; the value of this potential at the barrier is 12611 cm$^{-1}$. Thus, non-radiative de-excitation of the Rb* atom may add to its original kinetic energy of up to 165 cm$^{-1}$. It is worth noting that it will be ejected in the 5p$\,^2\Pi_{1/2}$ state, and not in the 5p$\,^2\Pi_{3/2}$ it was previously photo-excited to.

		This publication contains a extension of our combined experimental and theoretical investigation presented in the previous section. Here we focus on the formation of free RbHe-exciplex molecules from laser-excited Rb-doped He nanodroplets through the mechanism of electronic spin relaxation.
		
	\subsection*{Potassium-doped nanodroplets}
		Under the supervision of Nadine Halberstadt and me, a master research internship --- \emph{M2 Physique Fondamentale} --- titled ``\emph{\textbf{Dynamics of a superfluid helium nanodroplet doped with a single potassium atom}}'' has been performed by Maxime Martinez.

		The project investigates the static and dynamic behaviour of a single potassium (K) atom excited from the K-$^4$He$_{1000}$ equilibrium configuration to the K*(4p)-$^4$He$_{1000}$ and K*(5s)-$^4$He$_{1000}$ states. The choice of potassium was motivated by a discrepancy in the time-resolved experimental studies\citep{Schulz2001,Reho2000-1,Reho2000-2}. Moreover, the mass of potassium sits between those of the heavier alkalis like rubidium and cesium, and the lighter ones, like lithium and sodium. Therefore, potassium presents an interesting case, being on the borderline between the classical regime for heavy alkalis and a quantum--mechanical regime for the lighter ones. Both treatments of the equilibrium properties and the 5s$\leftarrow$4s excitation are studied. This work is not included in the thesis but can be found in \rf{Martinez2017}.
	
		It is concluded that quantum effects of K do exist but are not essential to the understanding and description of the dynamics. Therefore the K*(4p)-$^4$He$_{1000}$ excitation is studied with a classical description of K.

	\subsection*{Quantised vortices in droplets}
		The second aspect concerns a purely theoretical investigation that is inspired by recent work of Gomez and Vilesov \emph{et al}., where quantised vortices were visualised by doping He nanodroplets with silver atoms, subsequently ``soft landing'' them on a carbon screen. Electron-microscope images show long filaments of silver atom clusters that accumulated along the vortex cores. Also the formation of quantum-vortex lattices inside nanodroplets is evidenced by using X-ray diffractive imaging to visualise the characteristic Bragg patterns from xenon (Xe) clusters trapped inside the vortex cores.
		
		One of the most unambiguous signatures of the quantum mechanical nature of a substance---and indeed superfluidity---is the appearance of quantised vortices. In contrast to a normal fluid, which will rotate as a solid body when its container moves at low angular velocity, a superfluid will remain at rest. However, above a certain critical angular velocity the thermodynamically stable state of a superfluid includes one or more quantum vortices. Such a vortex can be characterised by a macroscopic wave function and quantised velocity circulation in units of $\kappa=\frac{h}{m}$, where $h$ is Planck’s constant and $m$ is the mass of the $^4$He atom\citep{Don91,Pit03}. Recently, the study of vorticity was extended to finite systems such as BECs confined to traps\citep{Pit03,Fetter2009}. The transfer of energy and angular momentum in finite systems between quantised vortices and surface excitations is of particular interest, as it defines the nucleation dynamics, shape, and stability of the involved vortices\citep{Pit03,Fetter2009}. In comparison to confined BECs, $^4$He droplets are self-contained and present a case for the strongly interacting superfluid. Moreover, the diameter of a vortex core which is approximately 0.2 nm in superfluid $^4$He\citep{Don91} is small relative to the droplet size, suggesting a three-dimensionality of the vortices in droplets. Vorticity in $^4$He droplets has therefore attracted considerable interest\citep{Clo98,Lehmann2003,Bar06,Sti06}.
	
		Recently, Gomez \emph{et al}. performed experiments\citep{Gom12} where vortices inside superfluid $^4$He droplets, produced by the expansion of liquid helium, were traced by introducing Ag atoms which clustered along the vortex lines, into the droplets. The helium droplets needed by these kind of experiments need to be larger than used before for single atom spectroscopy and dynamics studies because they need to be be big enough to be able to host an array of vortices, doped with many Ag clusters. A schematic of the experimental principle is shown in \fig{fig:vortex-machine}. Helium droplets are produced by expansion of He, at 20 bar and a temperature $T_0$=5.4--7 K, into vacuum through a nozzle. The droplets cool rapidly via evaporation and reach a temperature of 0.37 K\citep{Hartmann1995}, which is well below the superfluid transition temperature $T_\lambda=2.17\unit{K}$\citep{Don91,Pit03}. Further downstream, the droplets capture 10$^3$–10$^6$ Ag atoms in an oven\citep{Log11d}. The droplets are then collided against a thin carbon film substrate at room temperature\citep{Log11d}. Upon impact, the droplets evaporate, leaving on the surface the Ag traces, which are subsequently imaged via a transmission electron microscope (TEM). The prevalence of elongated track-shaped deposits (see \fig{fig:silver-filament}) shows that vortices are present in droplets larger than about $300\unit{nm}$ and that their lifetime exceeds a few milliseconds.
	
		Two years later Gomez \emph{et al}. reported\citep{Gom14} on the formation of quantum vortex lattices inside droplets. They used single-shot femtosecond X-ray coherent diffractive imaging to investigate the rotation of single, isolated superfluid helium-4 droplets containing about $10^8$--$10^{11}$ atoms. The formation of quantum vortex lattices inside the droplets was confirmed by observing the characteristic Bragg patterns from xenon clusters trapped in the vortex cores (see \fig{fig:vortex-array}).
	
	\subsection*{Head-on cllisions}
		Motivated by recent experiments that use Xe atoms to visualise vortex arrays in very large helium droplets\citep{Gom14,Jon16}, we present here a first step towards the description of the capture  of Xe atoms by helium droplets, namely head-on collisions of Xe atoms against a $^4$He$_{1000}$ droplet. A discussion on the dynamic capture of Xe atoms by droplets hosting vortex lines and vortex arrays will be provided by a forthcoming study combining DFT simulation of vortex arrays as in\rfs{Anc14,Anc15} for helium nanocylinders and nanodroplets  and collision with Xe atoms as in this work. Whenever possible, the results for Xe, a heliophilic atom,  are contrasted with results for Cs, a heliophobic atom with similar mass.
		
		We consider a droplet made of $N=1000$ helium atoms. Its ground state structure is obtained using DFT and gives a sharp-density  radius of about 22.2 \AA{}. Then the dynamics is initiated by placing the Xe atom 32 \AA{} away from the center of mass (COM) of the droplet with an impact parameter equal to zero (head-on collision). The simulations are carried out for initial Xe velocities  $v_0$ ranging from 200 to  600 m/s in the system of reference of the droplet, corresponding to kinetic energies between  315.8 K and  2842 K. These energies can be compared to the solvation energy of a Xe atom at the center of a $^4$He$_{1000}$ droplet, $S_{{\rm Xe}} = E({\rm Xe}@^4{\rm He}_{1000}) - E(^4{\rm He}_{1000}) = -316.3$ K. For the sake of comparison, the solvation energy of Cs is -5.2 K and its equilibrium position is  in a dimple at the outer droplet surface, about 26.6 \AA{} from its centre. 

		Thermal Xe atoms ($v_0 \sim$ 240 m/s) are used in the experiments\citep{Gom14,Jon16}, and the average droplet velocity is about 170 m/s\citep{Gom11}.

		We show that head-on collisions of helium nanodroplets with xenon, a heliophilic atom, involve a kinetic energy exchange of the same order of magnitude as cesium, a heliophobic atom with similar mass. In both cases, this energy is largely dissipated by producing energetic waves in the droplet or it is carried away by promptly emitted helium atoms. The difference between the two atoms is due to the different nature of their interaction with helium. Density build up is observed around the heliophilic xenon during the dynamics, whereas a bubble is created around the heliophobic cesium. Thus it takes a much higher velocity for xenon to go through the droplet and escape than for cesium, as could be expected.
	
	\subsection*{Capture by He droplets}
		Recently, a technique has been introduced to determine the size of large He droplets ($N> 10^5$). It is based on the attenuation of a continuous droplet beam through collisions with Ar atoms at room temperature\citep{Gom11}. The pickup chamber of the droplet beam apparatus is filled with argon gas and the helium droplets experience multiple, isotropic collisions with the Ar atoms on their way towards the detection chamber. 

		Large helium droplets could also be doped in this way. This method, using Xe atoms, has been instrumental for detecting and imaging quantized vortex arrays in helium droplets\citep{Gom14,Jones2016}. Xe atoms were used in these experiments because of their large sensitivity to the x-ray coherent diffractive imaging employed to detect them within the helium droplets. Experiments with large superfluid helium droplets are reviewed in a recent publication\citep{Tan17}.

		The impurity-droplet interaction in the presence of vortices is also relevant as the first stage of a more complex process leading to the formation of nanowires, see e.g.\rfs{Lebedev2011,Gom12,Lat14,Tha14}. Long filaments made of micrometer-sized solid hydrogen particles trapped on quantized vortex cores were used to directly image the vortex reconnection between quantized vortices in superfluid helium\citep{Bewley2008}.
		
		Here we present results obtained within TDDFT for the collision and capture of Xe and Ar atoms by a $^4$He$_{1000}$ droplet at different kinetic energies and impact parameters. Special attention is paid to the  time-dependent interaction of Xe and Ar atoms with helium nanodroplets hosting vortex lines, and to the effect of multiply-doped vortex arrays in large helium droplets.

		Due to the heavy computational cost of the TDDFT simulations presented here, we  address only a few facets of the capture process that we consider of experimental relevance rather than carrying out a systematic study of the process. In particular:
		\begin{itemize}
			\item We study the capture of Xe atoms by a $^4$He nanodroplet, both for head-on collisions and for different impact parameters,
with velocities ranging from thermal values up to several hundred m/s.
The results of peripheral collisions  
with different values of the impact parameter 
are used to estimate the cross section for the
Xe capture.
			\item We study how a Xe atom dynamically interacts with a 
droplet hosting a vortex line, under different initial conditions
resulting in different velocity regimes of the impurity as it 
collides with the vortex core:
			\begin{enumerate}
				\item[i)]  a Xe atom initially at rest on the droplet surface and
sinking under the effect of solvation forces;
				\item[ii)] a head-on collision of a moving Xe or Ar atom  against the
$^4$He nanodroplet.
			\end{enumerate}
			\item We study the stationary state of 
a large $^4$He$_{15000}$ 
droplet hosting a ring of six vortex lines, doped with 
Ar atoms completely filling all six vortex cores. This is the simplest system that mimics those
experimentally described in\rf{Gom14}, where 
doped vortex arrays embedded in rotating $^4$He microdroplets have been imaged.
		\end{itemize}
		

	\subsubsection*{Capture by vortex-free droplets}
	We have simulated head-on collisions of a Xe atom with a
$^4$He$_{1000}$ droplet at relative velocities $v_0$ 
ranging from 200 to 600 m/s. \fig{fig1-capture} 
displays two-dimensional plots of the helium density 
for the highest value, $v_0= 600$ m/s. This velocity is 
well above the range of velocities typically encountered
in experiments\citep{Gom11,Gom14,Jones2016}.  
In spite of the appearance of disconnected helium density shown in the 
$t= 87$ ps frame, we have found that the Xe atom eventually 
turns around and is 
captured again inside the droplet even at that relatively high impact velocity. 
Note that the Xe impurity, even when it temporarily emerges from the bulk of the 
droplet, appears to be coated with a few
$^4$He atoms, see the configuration at 87 ps.

	\subsubsection*{Vortex lines}
	To determine the structure of a droplet hosting a singly-quantized linear vortex 
we have started the imaginary time iteration from a helium  
density in which the vortex is ``imprinted''. For this purpose, a vortex line
along the $z$  can be described by the effective wave function 
%
\begin{equation}
\Psi_0(\mathbf{r}) =  \rho_0^{1/2}(r) \, e^{i  \, {\cal S}(\mathbf{r})} = \rho_0^{1/2}(\mathbf{r}) \, \frac{(x + i y)}{\sqrt{x^2 + y^2}} 
\label{eq11}
\end{equation}
%
where $\rho_0(\mathbf{r})$ is the density of 
either the pure or  the impurity-doped droplet without vortex.  
Vortex lines along other directions passing through a chosen point 
can be imprinted as well\citep{Pi07}.
%;  more involved imprinted configurations can be found {\it e.g.} in ref. \citep{Pi07}. 
 
In the case represented by \eq{eq11}, if the impurity is within the vortex
core along a symmetry axis of the impurity-droplet complex,
the effective wave function $\Psi_0({\mathbf r})$ -- before and after relaxation -- is an eigenvector of the  angular 
momentum operator  $\hat{L}_z = -i \; \hbar \partial/\partial \theta$. 
The angular momentum of the droplet is then
 %
\begin{equation}
\langle \hat{L}_z \rangle = \langle \Psi_0(\mathbf{r}) | \hat{L}_z  | \Psi_0(\mathbf{r}) \rangle = N \; \hbar
\label{eq12}
\end{equation}
%

	\subsubsection*{Capture dynamics by vortices}
	To study the interaction of an atomic impurity with vortices, 
we have imprinted a vortex line in the $^4$He$_{1000}$ droplet 
and prepared the Xe atom in different kinematic conditions. 
 
 The inelastic scattering of xenon atoms by quantized vortices in superfluid bulk helium has been addressed in\rf{Psh16}.
  It was found that
  a head-on collision leads to the capture of Xe by the vortex line for $v_0=$ 15.4 m/s, but not  for $v_0$=23.7 m/s.
We have carried out  an equivalent  simulation by initially placing
 the Xe atom  inside the droplet 10 \AA{} away
from the vortex line and   sending it  head-on towards the vortex at a velocity of 10 m/s.
This velocity  is of the order of the thermal velocity of a Xe atom in a droplet under experimental conditions,
once the droplet has thermalized after capturing the Xe atom  ($T \sim$0.4 K)\citep{Toe04}. 
Since the equilibrium position of  the Xe atom is at the center of the droplet, it  moves to this region and remains there during the rest of the simulation.
In this region of the droplet, the Xe atom is also attracted by the vortex, but it is deflected by the superfluid flow around the vortex line and ends up orbiting around it.
 Hence it is captured by the vortex without getting into its  core.
 
 	\subsubsection*{Vortex arrays in droplets}
 	The existence of ordered vortex lattices inside $^4$He droplets has been established  by the appearance of Bragg patterns from 
Xe clusters trapped inside the vortex cores  in droplets made of $N= 10^8 - 10^{11}$ atoms
(corresponding to radii from 100 to 1000 nm)\citep{Gom14,Jones2016}. We have 
recently studied the stability of vortex 
arrays made of up to $n_v=9$ vortices
inside a $^4$He nanodroplet using the DFT approach\citep{Anc15}.  
It was found that 
the energetically favored structure for $n_v > 6$ is a ring 
of vortices encircling a vortex at the center of the droplet.
Fot $n_v=6$,  the 
configuration with a six-vortex ring is found to have almost 
the same energy as the five-fold ring
plus a vortex at the center. The former structure 
has been experimentally observed\citep{Gom14,Jones2016,Ber17}, 
although classical vortex theory 
predicts for it a much higher free energy cost than for the latter\citep{Cam79}.
Similar equilibrium structures have been obtained within DFT for
helium nanocylinders hosting vortex arrays\citep{Anc14}.

We have looked for stationary configurations of a 6-vortex ring
in a rotating He$_{15000}$ droplet by solving the
EL equations in the corotating frame with a fixed
angular velocity. Each vortex core is filled with Ar
atoms, and the system is allowed to fully relax.
In the end, the column of atoms inside each vortex core reaches an equilibrium structure 
where the Ar atoms are separated by a distance which  is roughly that of the Ar dimer.
One such configuration is shown in \fig{fig13-capture}. Note that 
the vortex cores are almost straight lines, whereas in an
undoped droplet rotating with the same velocity 
the vortex lines would be bent, 
as shown  e.g. in \fig{fig7-capture}.
The Ar atoms are not shown in the Figure.
The localized structures appearing in the vortex cores are 
regions of highly inhomogeneous, high  $^4$He density
resulting from the Ar-He attractive potential.
	
	
		We show that Xe and Ar atoms at thermal velocities are readily captured by helium droplets, with a capture cross section similar to the geometric cross section of the droplet. Crucially for the subsequent capture of impurities by vortex lines, we have also shown that most of the kinetic energy of the impinging impurity is lost in the capture process during the first tens of picoseconds. This happens either by the ejection of prompt-emitted He atoms, or by the production of sound waves and large deformations in the droplet. 

		In addition, we have also shown that if the droplet hosts a vortex, slowly moving impurities are readily captured by the vortex line. Rather than being trapped inside the vortex core, the impurity is bound to move at a close distance around it. Besides the crucial energy loss when the impurity hits the droplet, the capture by the vortex is favoured by a further energy transfer from the impurity to the droplet: large amplitude displacements of the vortex line ---as shown in the ESI\citep{ESI}--- take place, constituting another source of the kinetic energy loss in the final stages of the capture. A related issue is the appearance of Kelvin modes in the vortex line, that is not only bent, but also twisted in the course of the collision.

		We can conclude that if the kinematic conditions of the collision (kinetic energy and impact parameter) lead to the capture of the impurity by the droplet, the pinball effect caused by the droplet surface can facilitate the meeting of the Xe/Ar atom and the vortex line ---and the possible capture of the atom by the vortex--- since both have a tendency to remain in the inner region of the droplet. We have shown this in the case of Xe at $v_0$=200~m/s: Xe is captured during its second transit across the droplet, whereas this could not have happened in bulk liquid helium\citep{Psh16}. This effect could explain the capture of impurities by vortex lines even in the very large droplets used in the observation of filament-shaped nanostructures.

	\subsection*{Future prospects}
	Our work on the real-time dynamics of the photo-excitation of an alkali metal atom on the surface of a helium nanodroplet is quite extensive. Conducting the same type of studies on other types of dopant species which are solvated more deeply inside He droplets (e.g. alkaline earth metals, transition metals) would give further insight into the mechanisms of desolvation and ejection of excited impurity atoms out of He nanodroplets\citep{Loginov:2007,Loginov:2012, Kautsch:2013,Lindebner:2014}.
	
	Moreover, a more complete description of the couplings between electronic states and the configurational degrees of freedom in such excited complexes induced by the He droplet environment would be highly desirable\citep{Closser:2014,Masson:2014}. In a recent advance, electronic relaxation of Ba$^+$ cations in He nanodroplets, based on a diabatisation of the He–Ba$^+$ ground and excited electronic states interaction potentials\citep{Vindel:2018}, has been proposed as a mechanism for ejecting Ba$^+$ and Ba$^+$He$_n$ off He droplets. These mechanisms for spin-relaxation and inter-electronic state relaxation have to be confirmed by real-time dynamics studies.
	
	Finally, the capabilities of the He-DFT approach might help elucidate processes of experimental interest, such as the capture of one or several impurities by large droplets hosting a vortex array and how several atomic impurities, impinging upon a rotating droplet hosting vortices, react to form small clusters, eventually being trapped within the vortex cores as shown by the appearance of filament-shaped nanostructures in experiments.
	
	In all these future lines of investigation, $^4$He-DFT and TDDFT will be essential tools, given their ability to accurately describe the equilibrium and dynamics properties of realistic size helium nanodroplets, possibly hosting vortices, in interaction with dopants. In this context it will be extremely interesting to couple $^4$He-TDDFT with quantum molecular dynamics in order to go beyond the mean field approximation of the dopant dynamics in the helium environment. A promising alternative in the correlated basis function (CBF) approach and its multi-component approach recently developed by Rader \textit{et al.}\citep{Rader2017}