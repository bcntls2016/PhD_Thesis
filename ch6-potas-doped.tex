\chapter{Supervised work: potassium-doped nanodroplets}
	\lettrine[lines=4]{\color{activeColor}U}{nder} the supervision of Nadine Halberstadt and me, a master research internship --- \emph{M2 Physique Fondamentale} --- titled ``\emph{\textbf{Dynamics of a superfluid helium nanodroplet doped with a single potassium atom}}'' has been performed by Maxime Martinez.

	The project investigates the static and dynamic behaviour of a single potassium (K) atom excited from the K-$^4$He$_{1000}$ equilibrium configuration to the K*(4p)-$^4$He$_{1000}$ and K*(5s)-$^4$He$_{1000}$ states. The choice of potassium was motivated by a discrepancy in the time-resolved experimental studies\citep{Schulz2001,Reho2000-1,Reho2000-2}. Moreover, the mass of potassium sits between those of the heavier alkalis like rubidium and cesium, and the lighter ones, like lithium and sodium. Therefore, potassium presents an interesting case, being on the borderline between the classical regime for heavy alkalis and a quantum--mechanical regime for the lighter ones. Both treatments of the equilibrium properties and the 5s$\leftarrow$4s excitation are studied. This work is not included in the thesis but can be found in \rf{Martinez2017}.
	
	It is concluded that quantum effects of K do exist but are not essential to the understanding and description of the dynamics. Therefore the K*(4p)-$^4$He$_{1000}$ excitation is studied with a classical description of K.
\clearpage{\pagestyle{empty}\cleardoublepage}