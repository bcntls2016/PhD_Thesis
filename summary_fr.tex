%\chapter{Résumé étendu en français}\markboth{\sectionfont\itshape\color{activeColor}Résumé étendu en français}{} %%% IN FRONTMATTER
%\chapter{Résumé étendu en français} %%% IN MAINMATTER
\chapter{Résumé étendu en français}\fancyhead[RE,LO]{\subsectionfont\itshape\color{activeColor}Résumé étendu en français} %%% IN BACKMATTER
	\lettrine[lines=4]{\color{activeColor}D}{ans} cette thèse, nous étudions deux aspects de la dynamique des impuretés atomiques interagissant avec des nanogouttelettes d'hélium superfluide (He), à savoir la photo-excitation des alcalis sur un nanogouttelettes et le dopage des nanogouttelettes contenant des tourbillons quantifiés avec des atomes de gaz rares. Pour les investigations théoriques, nous utilisons la théorie fonctionnelle He densité (He-DFT) et sa version dépendant du temps (He-TDDFT).

	Le premier aspect implique une collaboration expérimentale et théorique commune qui se concentre sur la photo-excitation du rubidium alcalin (Rb). Les alcalis sont une sonde très intéressante des gouttelettes de He car elles résident dans leur région de surface, où l'on a avancé que près de 100\% de condensation de Bose-Einstein pouvait être obtenue en raison d'une densité inférieure à celle du He superfluide.
	
	Le deuxième aspect concerne une investigation purement théorique inspirée des travaux récents de Gomez et Vilesov \emph{et al}., où les tourbillons quantifiés ont été visualisés en dopant des nanoparticules d'He avec des atomes d'argent, puis en les "atterrissant doucement" sur un carbone. écran. Les images au microscope électronique montrent de longs filaments d'amas d'atomes d'argent qui s'accumulent le long des noyaux des vortex. La formation de réseaux de tourbillons quantiques à l'intérieur de nanogouttelettes est également mise en évidence en utilisant l'imagerie par diffraction des rayons X pour visualiser les motifs de Bragg caractéristiques des amas de xénon (Xe) piégés à l'intérieur des noyaux de vortex.
	
	Nos simulations impliquant des gouttelettes hébergeant des tourbillons quantiques ouvrent la voie à d'autres investigations sur des gouttelettes hébergeant une série de vortex, impliquant de multiples impuretés.
		
	\section*{Introduction (section 1.3 de la thèse)}
		Jusqu'aux années 1980, la plupart des travaux expérimentaux et théoriques ont été effectués sur des systèmes en masse, c'est-à-dire des systèmes de l'ordre du nombre d'atomes $N_A$. Ce n'est qu'au cours des deux dernières décennies que les progrès technologiques ont permis aux expérimentateurs de créer des gouttelettes d'hélium superfluides de taille nanométrique. Dès le début des années 1990, les nanogouttelettes d'hélium superfluides sont devenues un champ d'étude actif, à la fois expérimentalement et théoriquement. 
		
		Les nanogouttelettes d'hélium sont considérées comme des systèmes modèles idéaux pour explorer l'hydrodynamique quantique dans des superfluides isolés et isolés. L'objectif principal a été l'évolution de leurs propriétés avec le nombre d'atomes dans le groupe, jusqu'à ce que la limite de matière condensée soit atteinte. Les amas d'hélium sont particulièrement intéressants dans la mesure où les effets quantiques jouent un rôle clé dans la détermination de leurs propriétés. En particulier, étant donné qu'un cluster d'hélium est un ensemble de bosons à environ 0.4~K\citep{Brink1990, Hartmann1995}, des manifestations de comportement collectif (comme la superfluidité) sont attendues. D'autre part, il n'est pas encore clair comment la taille finie d'un cluster affecte ce comportement collectif non-classique (ou dégénéré).

		Récemment, Toennies \emph{et al}.\citep {Hartmann1996} ont mesuré le spectre électronique de molécules de glyoxal noyées dans des amas He et l'ont trouvé cohérent avec une simulation théorique calculée en utilisant la courbe de dispersion de phonons de He II en vrac superfluide. Les auteurs eux-mêmes soulignent cependant qu'à la taille moyenne des clusters de 5500 atomes de He rapportée dans\rf{Hartmann1996}, les groupes sont si grands que les effets de taille finie dans la région intérieure sont négligeables (voir aussi\rfs{RamaKrishna1990-1,Chin1992}). Il n'est donc pas surprenant qu'ils trouvent des résultats cohérents avec le cas en vrac, en particulier pour une molécule facilement solvatée à l'intérieur de la grappe, pour laquelle les effets de surface jouent un rôle mineur. Par conséquent, l'influence de la taille des clusters He sur la superfluidité n'a pas été détectée jusqu'à présent.

		L'interaction hélium-hélium est déjà faible dans l'hélium liquide en vrac et dans les systèmes auto-liés finis tels que les gouttelettes, elle est encore plus faible, par ex. l'énergie de liaison par atome est $<$7.17~K. A cause de cela, les gouttelettes d'hélium se refroidissent très rapidement à cause de l'évaporation rapide et atteignent ainsi leur température limite d'environ 0.38~K en microsecondes. Les gouttelettes d'hélium pur sont des systèmes neutres et leurs propriétés comme leur taille, leur énergie de liaison et leurs spectres d'excitation ne sont pas faciles à déterminer expérimentalement et sont généralement obtenues par des méthodes indirectes. Cela n'a pas empêché les théoriciens de décrire des gouttelettes dopées $^4$He$_N$ en utilisant une grande variété d'approches en fonction de la taille et du caractère des gouttelettes allant de Quantum Monte Carlo, Hypernetted-Chain/Euler-Lagrange\citep{Krotscheck2001}, Variation Monte Carlo\citep{Gartner2018} et beaucoup d'autres.

		Une propriété clé de gouttelettes d'hélium, contrairement à l'hélium en vrac, est leur capacité à ramasser n'importe quel type de dopants avec lesquels ils entrent en collision. En fonction de la force de l'interaction dopant - $^4$He et de la tension superficielle de la gouttelette, on peut définir un paramètre sans dimension $\lambda$\citep{Anc95} avec une valeur critique $\lambda_0\sim 1.9$. Au-dessous de $\lambda_0$, les impuretés sont liées à la surface de la gouttelette (par exemple les alcalis) et, au-dessus, elles sont solvatées à l'intérieur de la gouttelette. Les gouttelettes peuvent donc être dopées avec presque toutes sortes d'espèces atomiques ou moléculaires.

		Du point de vue de la gouttelette, cela signifie qu'il est possible d'utiliser les dopants comme des sondes douces pour déterminer les propriétés superfluides des gouttelettes d'hélium qui seraient inaccessibles avec d'autres méthodes. Pour deux exemples de ceci, voir\rfs{Gre98,Sin89,RamaKrishna1990-2}, où un dopant est utilisé pour sonder le caractère superfluide de petites gouttelettes $^4$He et\rfs{Hartmann1999, Harms1999} pour voir leurs températures limites.
		
		De plus, du point de vue des impuretés, il permet un large spectre d'études expérimentales possibles. Du fait que les gouttelettes d'hélium sont des liquides superfluides ultra-froids, et permettent donc une grande mobilité de tout dopant prélevé, on peut réaliser des études de spectroscopie à haute résolution. En contrôlant finement le nombre de dopants prélevés [29], on peut utiliser des gouttelettes comme matrice pour créer des structures auto-organisatrices de molécules polaires, ou des amas de métaux très froids et étudier leur explosion de Coulomb.

		L'une des propriétés les plus intrigantes des gouttelettes d'hélium superfluides est le fait qu'elles peuvent héberger des vortex quantifiés. En raison de leur température ultra basse, ils sont de véritables liquides quantiques et leur tourbillon et leur moment cinétique sont quantifiés. L'existence de tourbillons quantifiés a été anticipée car ils ont été créés et observés dans des BEC constitués de gaz dilués. Cependant, la détection de tourbillons quantifiés est encore expérimentalement difficile (voir \ scn {sec: quant-vort} dans cette thèse).

		Beaucoup de travail a été fait sur les gouttelettes d'hélium au cours des dernières décennies, à la fois expérimentalement et théoriquement. A partir des spectres d'absorption de gouttelettes d'hélium dopées par des métaux alcalins, l'étude des gouttelettes dopées mixtes $^3$He--$^4$He gouttelettes, électrons dans de l'hélium liquide, à l'étude de la vitesse critique de Landau à l'intérieur de petites $^4$He gouttelette. Pour un aperçu complet du travail effectué au cours des deux dernières décennies, le lecteur intéressé est renvoyé aux documents de révision dans\rfs{Barranco2006,Ancilotto2017,Mudrich2014}.
	
	\section*{Méthodes utilisées (sections. 2.1, 2.2)}
		D'un point de vue théorique, l'hélium superfluide doit être considéré comme un système quantique de grande dimension. Les calculs quantiques Monte Carlo \ citep {Kro02} (QMC) et quantique direct \ citep {deL06, deL10, Agu13} sont les méthodes les plus précises, mais leur demande dépasse rapidement les ressources informatiques actuellement disponibles lorsque le nombre d'atomes d'hélium augmente. De plus, QMC ne peut pas décrire l'évolution dynamique de l'hélium superfluide en temps réel. Pour pallier ces limitations, des méthodes semi-empiriques basées sur le formalisme de la DFT ont été introduites: \ citep {Str87a, Str87b, Dal95}. DFT peut être appliqué à des systèmes beaucoup plus grands que QMC et permet une formulation dépendant du temps. En tant que tel, il offre un bon compromis entre précision et faisabilité computationnelle. Le principal inconvénient du DFT est que la fonction énergétique exacte n'est pas connue et doit donc être construite de manière semi-empirique. De plus, les gouttelettes d'hélium dopées sont limitées à une description du champ moyen de l'interaction dopant-hélium. Néanmoins, la DFT est la seule méthode à ce jour capable de reproduire avec succès les résultats d'une vaste gamme d'expériences résolues en temps dans l'hélium superfluide, pour des tailles réalistes par rapport aux conditions expérimentales.

		Le point de départ de la méthode fonctionnelle de densité est le théorème de Hohenberg-Kohn\citep{Hohenberg1964} (HK), qui indique que l'énergie de l'état fondamental $E_v$ d'un système \emph{inhomogène inhomogène} dans un potentiel statique $v$ peut être écrit comme une fonctionnelle unique de la densité d'un corps $\rho$ as $F[\rho]$; un universel fonctionnel ---valable pour \emph{any} nombre de particules et \emph{any} potentiel externe $v$--- de la densité.
		
		Kohn et Sham (KS) ont ensuite reformulé\citep{Kohn1965} la théorie en introduisant un schéma d'approximation pour le $F[\rho]$ fonctionnel qui est analogue à la méthode de Hartree, mais contient également la majeure partie des effets de corrélation inhérents à l'interaction. systèmes à plusieurs corps. L'approximation commence par scinder la fonctionnelle en une énergie cinétique et une partie d'énergie de corrélation. L'énergie cinétique est celle d'un système fictif de particules \emph{non-interactives} de densité $\rho$. La partie de corrélation correspond à un système \emph{interagissant} avec la même densité. Pour la partie cinétique, cela nous permet d'écrire l'énergie cinétique totale comme la somme des énergies cinétiques individuelles des particules non-interactives. Il y a une différence entre la véritable énergie cinétique du système en interaction et celle du système fictif, en raison de la négligence des corrélations. Cette différence est corrigée et prise en compte dans la partie énergie de corrélation. C'est seulement la somme des deux qui donne l'énergie d'état fondamental correcte du système de particules en interaction.
		
		Parce que la fonction que nous avons utilisée dans ce travail est calibrée pour produire le bon comportement de l'hélium liquide en vrac à température nulle et sans pression, nous supposons une condensation complète de l'hélium entre Bose-Einstein (BE). Dans ce cas, tous les atomes d'hélium occupent le même état fondamental KS-orbitale. Par conséquent, la fonction d'onde à plusieurs corps et la densité simplifient davantage et permettent de décrire l'ensemble du condensat en définissant une fonction d'onde efficace qui ne dépend que d'une coordonnée dans un espace cartésien à trois dimensions.
		
		Le travail difficile consiste à concevoir une fonction telle que les propriétés physiques souhaitées de l'hélium puissent être récupérées. C'est loin d'être trivial mais plusieurs de ces fonctionnelles de densité sont disponibles maintenant. La densité fonctionnelle utilisée dans ce travail est basée sur la fonctionnelle de densité \ emph {Orsay-Trento} qui est discutée dans \ scn {sec:otdft} de la thèse. Il utilise une approche non locale à échelle finie et est, à ce jour, le modèle le plus précis dans la mesure où ses paramètres ont été ajustés pour reproduire les propriétés globales de l'hélium liquide à température nulle.
		
		En présence de densités de liquide hautement inhomogènes, par ex. impuretés atomiques avec une interaction He-X très forte, la fonction OT devient numériquement instable. Pour résoudre ce problème, un terme de pénalité d'énergie supplémentaire est imposé. L'inclusion de ce terme dans la fonction OT empêche l'accumulation excessive de densité. La suppression des termes non locaux de la fonction OT d'origine et l'ajout du terme de pénalité donnent une fonction de densité modifiée qui est appelée \ emph {Solid Functional}. Voir \ ref {sec: solide} pour plus de détails et ses paramètres.
		
		Pour décrire l'évolution temporelle du système, le théorème de Runge-Gross étend DFT à sa version dépendant du temps TDDFT \ citep {Run84}. La variation fonctionnelle d'une action associée (voir \ eq {eq: action dans la thèse} pour un exemple) conduit à une équation d'Euler-Lagrange (EL) dépendant du temps. En considérant seulement les états stationnaires du Hamiltonien, on obtient des équations EL indépendantes du temps qui, lors de la résolution, donnent l'énergie de l'état fondamental du système.
		
		Pour plus de détails sur la façon dont les calculs statiques et dynamiques sont résolus pour les différentes impuretés dans les potentiels d'interactions isotropes et anisotropes, veuillez vous reporter aux Section~2.3 et Section~2.4.
		
	\section*{Dynamique d'état excité des nanogouttes dopées aux alcalis}
		Dans un article de 1996 \ citep {Griffin1996}, Griffin et Stringari ont fait valoir que près de 100\% de condensation de Bose-Einstein pouvait être obtenue dans la région superficielle à faible densité de He superfluide à $ T = 0 $, contre seulement 10\% dans le vrac. Il est donc évident qu'une sonde à perturbation minimale capable d'étudier la surface d'un cluster He est très souhaitable.
		
		Il a été soutenu d'un point de vue théorique \ citep {Dalfovo1994} que les atomes alcalins résident sur la surface de la grappe. Des preuves expérimentales ont été trouvées \ citep {Stienkemeier1995-1, Stienkemeier1995-2, Ancilotto1995-1} plus tard quand il a été observé que le spectre de fluorescence induite par laser (LIF) du sodium a été déplacé par rapport au sodium dans la phase gazeuse en raison de la présence du cluster He. Cependant, pas autant que les atomes alcalins dans la masse de l'hélium liquide.
		
		Il n'est donc pas surprenant que les atomes alcalins soient un choix très naturel pour exactement ce type d'études. Par exemple, avec un paramètre de solvatation (voir \ scn {sec: helium-droplets}) de $ \ lambda = 0.729 $ \ citep {Anc95}, Rb restera lié à la surface de la gouttelette. De plus, les alcalis ont un spectre d'absorption simple et bien connu. De plus, leur structure électronique simple à une valence permet une modélisation théorique détaillée. Ils n'introduisent que des perturbations faibles (les énergies d'interaction alcali-hélium sont de l'ordre de 1 cm $ ^ {- 1} $ \ citep {Pat91}). Enfin, des calculs théoriques \ citep {Ancilotto1995-2, Kanorsky1994} et des spectres expérimentaux \ citep {Tabbert1995, Takahashi1993, Beijersbergen1993} d'atomes alcalins dans l'hélium liquide en vrac sont disponibles à titre de comparaison.
		
		Etant donné que les alcalis sont des objets idéaux pour sonder la région limite des nanogouttelettes, le $ n \ mathrm {p} \, ^ 2 \ mathrm {P} \! \ Longleftarrow \! N \ mathrm {s} \, ^ 2 \ mathrm Les transitions {S} $ des atomes alcalins ont suscité beaucoup d'intérêt d'un point de vue expérimental aussi bien que théorique. La spectroscopie des états excités supérieurs a été complètement explorée \ citep {Log11b, Log11a, Lackner2012, Lackner2013, The11, Fec12, Pif10, Lac11, Theisen2011, Lac13}. Les spectres obtenus peuvent être reproduits avec succès par un modèle pseudo-diatomique, à l'exception des états excités supérieurs, où le modèle échoue progressivement en raison des limitations imposées par son domaine de validité \ citep {Sti96, Bunermann2007}. Alors que l'effet des états excités sur les spectres est maintenant assez bien compris, leur influence sur les dynamiques suivantes est largement inexplorée.
		
		Dans cette partie de la thèse, les résultats de la dynamique en temps réel d'un seul atome de rubidium (Rb) électroniquement excité résidant dans la fossette de surface d'une nano-gouttelette d'hélium sont présentés. L'atome est excité de son état fondamental 5s$\,^2\Sigma_{1/2}$ aux 5p$\,^2\{\Sigma,\Pi\}$ et 6p$\,^2\{\Sigma,\Pi\}$ manifold (voir \ scn {sec: dim-model} pour une explication des étiquettes d'état électroniques utilisées). Habituellement, ils désorbent lors de l'excitation soit comme un atome nu ou comme un complexe avec un ou plusieurs atomes d'hélium, appelé un "exciplex".	
	
	\section*{Imager la dynamique des états excités}	
		L'article suivant est une étude combinée expérimentale et théorique centrée sur l'imagerie et la caractérisation de la dynamique suivant les excitations 5p $ \ leftarrow $ 5s et 6p $ \ leftarrow $ 5s de rubidium hébergé par une nanogouttelette d'hélium. L'expérience a utilisé des techniques pompe-sonde femtoseconde avec un premier laser excitant le Rb sur la surface des gouttelettes au temps $ t_ {exc} $ et un second laser l'ionisant pour la détection avec VMI au temps $ t_ {ion} $. Les résultats ont caractérisé un délai critique, appelé «temps de repli», entre deux résultats opposés. Si $ t_ {ion} -t_ {exc} \ leq \ tau $, l'atome Rb sortant est encore assez proche de la gouttelette lorsque le laser sonde rend son interaction attrayante. Par conséquent, le Rb $ ^ + $ se retourne et est solvaté. D'autre part, pour $ t_ {ion} -t_ {exc} \ geq \ tau $, l'ionisation se produit trop tard pour que Rb $ ^ + $ ressente une attraction appréciable de la gouttelette, et il y avait déjà trop d'énergie cinétique, pour qu'il s'échappe.
		
		L'étude théorique a porté sur la compréhension de la dynamique de désorption et la détermination des temps de repli à comparer avec l'expérience. Il a fait usage du He-TDDFT présenté dans \ scn {sec: td-dft}, à la fois dans les états excités et ionisés. Les résultats sont présentés dans l'article suivant qui a été publié dans le Journal of Physical Chemistry Letters \ citep {Vangerow2017}.
		
		Dans nos simulations, nous trouvons que les états excités au collecteur 5p et 6p désorbent à des échelles de temps très différentes, séparées par 2 ordres de grandeur ($ \ sim $ 100 ps et $ \ sim $ 1 ps pour respectivement 5p et 6p). Ceci est en bon accord avec les résultats expérimentaux où le comportement de désorption des atomes de Rb photo-excités est déterminé en utilisant un schéma pompe-sonde femtoseconde. 

	\section*{Dynamique de désorption des exciplexes RbHe}
		Cependant, dans nos simulations, l'excitation à l'état $ 5 \ mathrm {p} \, ^ 2 \ Pi_ {3/2} $ aboutit à un exciplexe de RbHe lié à la surface, contrairement au cas expérimental où l'exciplexe RbHe désorbe des gouttelettes surface. En introduisant la relaxation de spin $ ^ 2 \ Pi_ {1/2} \ leftarrow {^ 2} \ Pi_ {3/2} $ dans les simulations, l'exciplex RbHe est capable de désorber à partir de la surface de la gouttelette, ce qui résout cette contradiction.
		
		Comment résoudre la divergence entre l'observation expérimentale que les atomes de Rb, excités à l'état 5p $ \, ^ 2 \ Pi_ {3/2} $, se détachent de la surface des gouttelettes, et les simulations TD-DFT qui montrent qu'elles aboutissent à un état lié à la surface? C'est la question qui a conduit à ce travail. Lors de la photo-excitation de Rb à l'état 5p $ \, ^ 2 \ Pi_ {3/2} $, un atome He peut y être attaché formant un exciplex HeRb; cela ne peut pas arriver si Rb est excité à l'état 5p $ \, ^ 2 \ Pi_ {1/2} $ car il trouve une barrière (voir \ fig {fig: potentialals}) empêchant la formation d'exciplex.
		
		Dans la phase gazeuse, un exciplex HeRb 5p $ \, ^ 2 \ Pi_ {1/2} $ peut être formé s'il y a assez d'énergie cinétique pour que Rb * surmonte la barrière de potentiel; alternativement, la collision de HeRb 5p $ \, ^ 2 \ Pi_ {3/2} $ exciplex avec un autre atome ou complexe pourrait relâcher l'atome Rb * de la 5p $ \, ^ 2 \ Pi_ {3/2} $ à l'état 5p $ \, ^ 2 \ Pi_ {1/2} $, surmontant la barrière car les puits potentiels pour les deux états sont à des distances Rb-He similaires. Dans la phase condensée (gouttelettes) à la température de 0,4 K, aucun de ces mécanismes n'est disponible pour expliquer la formation des exciplexes HeRb 5p $ \, ^ 2 \ Pi_ {1/2} $ et leur éjection potentielle.
		
		Cependant, une autre possibilité pour cela est la désexcitation non radiative de 5p $ \, ^ 2 \ Pi_ {3/2} $ vers les 5p $ \, ^ 2 \ Pi_ {1/2} $ qui peuplent le dernier état et laisse l'atome Rb * avec assez d'énergie cinétique pour être éjecté. Notez de \ fig {fig: potentiels} que le minimum du potentiel 5p $ \, ^ 2 \ Pi_ {3/2} $ est de 12683 cm $ ^ {- 1} $, et celui du 5p $ \, ^ 2 \ Pi_ {1/2} $ potentiel est à 12518 cm $ ^ {- 1} $; la valeur de ce potentiel à la barrière est de 12611 cm $ ^ {- 1} $. Ainsi, une désexcitation non radiative de l'atome Rb * peut ajouter à son énergie cinétique d'origine jusqu'à 165 cm $ ^ {- 1} $. Il est à noter qu'il sera éjecté dans l'état 5p $ \, ^ 2 \ Pi_ {1/2} $, et non dans le 5p $ \, ^ 2 \ Pi_ {3/2} $ il était précédemment photo- excité à.
		
		Cette publication contient une extension de notre recherche expérimentale et théorique combinée présentée dans la section précédente. Nous nous intéressons ici à la formation de molécules de RbHe-exciplex libres à partir de nanoparticules de He dopées au Rb et excitées par laser grâce au mécanisme de relaxation de spin électronique.		

	\section*{Nanogouttelettes de potassium dopé.}
		Sous la supervision de Nadine Halberstadt et de moi, un master de recherche de recherche ---\emph{M2 Physique Fondamentale}--- intitulé ``\emph{\textbf{Dynamique d'un nanodroplet d'hélium superfluide dopé avec un seul atome de potassium}}'' a été joué par Maxime Martinez.
		
		Le projet étudie le comportement statique et dynamique d'un seul atome de potassium (K) excité de la configuration d'équilibre K-$^4$He$_{1000}$ au K*(4p)-$^4$He$_{1000}$ et K*(5s)-$^4$He$_{1000}$ états. Le choix du potassium a été motivé par une discordance dans les études expérimentales résolues dans le temps\citep{Schulz2001,Reho2000-1,Reho2000-2}. De plus, la masse de potassium se situe entre celles des alcalis plus lourds comme le rubidium et le césium, et les plus légères, comme le lithium et le sodium. Par conséquent, le potassium présente un cas intéressant, étant à la limite entre le régime classique pour les alcalis lourds et un régime quantique pour les plus légers. Les deux traitements des propriétés d'équilibre et de l'excitation 5s $\leftarrow$ 4s sont étudiés. Ce travail n'est pas inclus dans la thèse mais peut être trouvé dans\rf{Martinez2017}.
		
		On conclut que les effets quantiques de K existent mais ne sont pas essentiels à la compréhension et à la description de la dynamique. Donc l'excitation K*(4p)-$^4$He$_{1000}$ est étudiée avec une description classique de K.

	\section*{Tourbillons quantifiés en gouttelettes}
		Le deuxième aspect concerne une investigation purement théorique inspirée des travaux récents de Gomez et Vilesov \ emph {et al}., Où les tourbillons quantifiés ont été visualisés en dopant des nanoparticules d'He avec des atomes d'argent, puis en les "atterrissant doucement" sur un carbone. écran. Les images au microscope électronique montrent de longs filaments d'amas d'atomes d'argent qui s'accumulent le long des noyaux des vortex. La formation de réseaux de tourbillons quantiques à l'intérieur de nanogouttelettes est également mise en évidence en utilisant l'imagerie par diffraction des rayons X pour visualiser les motifs de Bragg caractéristiques des amas de xénon (Xe) piégés à l'intérieur des noyaux de vortex.
		
		L'une des signatures les plus claires de la nature quantique d'une substance - et en fait de la superfluidité - est l'apparition de vortex quantifiés. Contrairement à un fluide normal, qui tournera comme un corps solide lorsque son récipient se déplace à une faible vitesse angulaire, un superfluide restera au repos. Cependant, au-dessus d'une certaine vitesse angulaire critique, l'état thermodynamiquement stable d'un superfluide comprend un ou plusieurs vortex quantiques. Un tel vortex peut être caractérisé par une fonction d'onde macroscopique et une circulation de vitesse quantifiée en unités de $ \ kappa = \ frac {h} {m} $, où $ h $ est la constante de Planck et $ m $ est la masse de $ ^ 4 $ He atom \ citep {Don91, Pit03}. Récemment, l'étude de la vorticité a été étendue à des systèmes finis tels que les BEC confinés aux pièges \ citep {Pit03, Fetter2009}. Le transfert d'énergie et de moment cinétique dans les systèmes finis entre les tourbillons quantifiés et les excitations de surface est particulièrement intéressant car il définit la dynamique de nucléation, la forme et la stabilité des vortex impliqués \ citep {Pit03, Fetter2009}. En comparaison avec les BEC confinés, les gouttelettes $ ^ 4 $ He sont autonomes et présentent un cas pour le superfluide fortement interactif. De plus, le diamètre d'un noyau vortex d'environ 0,2 nm dans le superfluide $ ^ 4 $ He \ citep {Don91} est petit par rapport à la taille des gouttelettes, suggérant une tridimensionnalité des vortex dans les gouttelettes. Le tourbillon en $ ^ 4 $ He gouttelettes a donc attiré un intérêt considérable \ citep {Clo98, Lehmann2003, Bar06, Sti06}.
		
		Récemment, Gomez \emph{et al}. ont fait des expériences\citep{Gom12} où des tourbillons à l'intérieur des gouttelettes superfluides $^4$He, produites par l'expansion de l'hélium liquide, ont été tracés en introduisant des atomes d'Ag qui se groupaient le long des lignes de vortex dans les gouttelettes. Les gouttelettes d'hélium nécessaires à ce type d'expériences doivent être plus grandes qu'avant pour la spectroscopie atomique simple et les études de dynamique car elles doivent être suffisamment grandes pour pouvoir héberger une série de vortex dopés avec de nombreux agrégats d'Ag. Un schéma du principe expérimental est montré dans \fig{fig:vortex-machine}. Les gouttelettes d'hélium sont produites par expansion de He, à 20 bars et à une température $T_0$=5.4-7~K, dans le vide à travers une buse. Les gouttelettes refroidissent rapidement par évaporation et atteignent une température de 0.37~K\citep{Hartmann1995}, ce qui est bien en dessous de la température de transition superfluide $T_\lambda=2.17\unit{K}$\citep{Don91,Pit03}. Plus loin en aval, les gouttelettes capturent 10 atomes de carbone dans un four \citep{Log11d}. Les gouttelettes sont ensuite abordées contre un substrat de film de carbone mince à température ambiante \citep{Log11d}. Lors de l'impact, les gouttelettes s'évaporent, laissant sur la surface les traces d'Ag, qui sont ensuite visualisées par un microscope électronique à transmission (TEM). La prévalence des dépôts allongés en forme de piste (voir \fig{fig: silver-filament}) montre que les tourbillons sont présents dans des gouttelettes de plus de 300~nm et que leur durée de vie dépasse quelques millisecondes.
		
		Deux ans plus tard Gomez \ emph {et al}. rapporté \ citep {Gom14} sur la formation de réseaux de vortex quantique à l'intérieur des gouttelettes. Ils ont utilisé l'imagerie par diffraction de rayons X femtoseconde à une seule prise pour étudier la rotation de gouttelettes d'hélium-4 superfluides isolées, isolées, contenant environ 10$^8$-10$^{11}$ atomes. La formation de réseaux de vortex quantique à l'intérieur des gouttelettes a été confirmée en observant les patrons de Bragg caractéristiques des amas de xénons piégés dans les noyaux des vortex (voir \ fig {fig: vortex-array}).
	
	\section*{Collisions frontales}
		Motivé par des expériences récentes utilisant des atomes Xe pour visualiser des réseaux de vortex dans de très grandes gouttelettes d'hélium \ citep {Gom14, Jon16}, nous présentons ici un premier pas vers la description de la capture d'atomes de Xe par des gouttelettes d'hélium. Xe atomes contre une gouttelette $ ^ 4 $ He $ _ {1000} $. Une discussion sur la capture dynamique des atomes Xe par des gouttelettes hébergeant des vortex et des vortex sera fournie par une prochaine étude combinant la simulation DFT des réseaux vortex comme dans \ rfs {Anc14, Anc15} pour les nanocylindres et nanogouttelettes d'hélium et la collision avec les atomes Xe dans ce travail. Dans la mesure du possible, les résultats pour Xe, un atome héliophile, sont mis en contraste avec les résultats pour Cs, un atome héliophobe de masse similaire.
		
		Nous considérons une gouttelette faite de $ N = 1000 $ atomes d'hélium. Sa structure d'état fondamental est obtenue en utilisant DFT et donne un rayon de densité d'environ 22,2 \ AA {}. Ensuite, la dynamique est initiée en plaçant l'atome Xe 32 \ AA {} à l'écart du centre de masse (COM) de la gouttelette avec un paramètre d'impact égal à zéro (collision frontale). Les simulations sont effectuées pour des vitesses Xe initiales $ v_0 $ allant de 200 à 600 m / s dans le système de référence de la gouttelette, correspondant à des énergies cinétiques comprises entre 315,8 K et 2842 K. Ces énergies peuvent être comparées à l'énergie de solvatation de un atome Xe au centre d'une gouttelette $ ^ 4 $ He $ _ {1000} $, $ S _ {{\ rm Xe}} = E ({\ rm Xe} @ ^ 4 {\ rm He} _ {1000} ) - E (^ 4 {\ rm He} _ {1000}) = -316.3 $ K. Par souci de comparaison, l'énergie de solvatation de Cs est de -5.2 K et sa position d'équilibre est dans une fossette à la surface extérieure des gouttelettes. , environ 26.6 \ AA {} de son centre.
		
		Les atomes thermiques Xe ($ v_0 \ sim $ 240 m / s) sont utilisés dans les expériences \ citep {Gom14, Jon16}, et la vitesse moyenne des gouttelettes est d'environ 170 m / s \ citep {Gom11}.
		
		Nous montrons que les collisions frontales de nanogouttelettes d'hélium avec du xénon, un atome héliophile, impliquent un échange d'énergie cinétique du même ordre de grandeur que le césium, un atome héliophobe de masse similaire. Dans les deux cas, cette énergie est largement dissipée en produisant des ondes énergétiques dans la gouttelette ou elle est emportée par des atomes d'hélium rapidement émis. La différence entre les deux atomes est due à la nature différente de leur interaction avec l'hélium. Une accumulation de densité est observée autour du xénon héliophile lors de la dynamique, alors qu'une bulle est créée autour du césium héliophobe. Il faut donc beaucoup plus de vitesse pour que le xénon traverse la gouttelette et s'échappe que pour le césium, comme on pouvait s'y attendre.
	
	\section*{Capture par He gouttelettes}
		Récemment, une technique a été introduite pour déterminer la taille des grosses gouttelettes He ($ N> 10 ^ 5 $). Il est basé sur l'atténuation d'un faisceau continu de gouttelettes par des collisions avec des atomes d'Ar à température ambiante \ citep {Gom11}. La chambre de prélèvement de l'appareil à faisceau de gouttelettes est remplie de gaz argon et les gouttelettes d'hélium subissent de multiples collisions isotropes avec les atomes Ar sur leur chemin vers la chambre de détection.
		
		De grosses gouttelettes d'hélium pourraient également être dopées de cette manière. Cette méthode, utilisant des atomes Xe, a été instrumentale pour la détection et l'imagerie des réseaux de vortex quantifiés dans les gouttelettes d'hélium \ citep {Gom14, Jones2016}. Des atomes Xe ont été utilisés dans ces expériences en raison de leur grande sensibilité à l'imagerie par diffraction cohérente aux rayons X utilisée pour les détecter dans les gouttelettes d'hélium. Des expériences avec de grandes gouttelettes d'hélium superfluides sont passées en revue dans une publication récente \ citep {Tan17}.
		
		L'interaction impureté-gouttelette en présence de vortex est également pertinente en tant que première étape d'un processus plus complexe conduisant à la formation de nanofils, voir par exemple \ rfs {Lebedev2011, Gom12, Lat14, Tha14}. Des filaments longs constitués de particules d'hydrogène solides de taille micrométrique piégées sur des noyaux vortex quantifiés ont été utilisés pour imager directement la reconnexion vortex entre les vortex quantifiés dans l'hélium superfluide \ citep {Bewley2008}.
		
		Ici nous présentons les résultats obtenus dans TDDFT pour la collision et la capture des atomes Xe et Ar par une gouttelette $ ^ 4 $ He $ _ {1000} $ à différentes énergies cinétiques et paramètres d'impact. Une attention particulière est accordée à l'interaction dépendant du temps des atomes de Xe et Ar avec des nanogouttelettes d'hélium contenant des lignes de vortex, et à l'effet de réseaux de tourbillons dopés à plusieurs couches dans de grosses gouttelettes d'hélium.
		
		En raison du coût de calcul élevé des simulations TDDFT présentées ici, nous abordons seulement quelques facettes du processus de capture que nous considérons comme pertinentes expérimentalement plutôt que d'effectuer une étude systématique du processus. En particulier:
		\begin{itemize}
			\item Nous étudions la capture d'atomes de Xe par une nanogoutte de $^4$He, à la fois pour des collisions frontales et pour différents paramètres d'impact, avec des vitesses allant de valeurs thermiques allant jusqu'à plusieurs centaines de m / s. Les résultats des collisions périphériques avec différentes valeurs du paramètre d'impact sont utilisés pour estimer la section efficace pour la capture Xe.
			\item Nous étudions comment un atome Xe interagit dynamiquement avec une gouttelette hébergeant une ligne de vortex, dans différentes conditions initiales résultant en différents régimes de vitesse de l'impureté lorsqu'il entre en collision avec le noyau vortex:
			\begin{enumerate}
				\item[i)] un atome Xe initialement au repos sur la surface des gouttelettes et s'enfonçant sous l'effet des forces de solvatation;
				\item[ii)] une collision frontale d'un atome Xe ou Ar en mouvement contre la nanodroplet de $^4$He.
			\end{enumerate}
			\item Nous étudions l'état stationnaire d'une grosse gouttelette de $^4$He$_{15000}$ contenant un anneau de six lignes de vortex, dopées avec des atomes d'Ar remplissant complètement les six noyaux de vortex. C'est le système le plus simple qui imite ceux décrits expérimentalement dans\rf{Gom14}, où des réseaux de vortex dopés incorporés dans des microgouttes rotatives $^4$He ont été imagés.
		\end{itemize}

		\subsection*{Capture by vortex-free droplets}
			We have simulated head-on collisions of a Xe atom with a $^4$He$_{1000}$ droplet at relative velocities $v_0$ ranging from 200 to 600 m/s. \fig{fig1-capture} displays two-dimensional plots of the helium density for the highest value, $v_0= 600$ m/s. This velocity is well above the range of velocities typically encountered in experiments\citep{Gom11,Gom14,Jones2016}. In spite of the appearance of disconnected helium density shown in the $t=87$ ps frame, we have found that the Xe atom eventually turns around and is captured again inside the droplet even at that relatively high impact velocity. Note that the Xe impurity, even when it temporarily emerges from the bulk of the droplet, appears to be coated with a few $^4$He atoms, see the configuration at 87 ps.

		\subsection*{Vortex lines}
			To determine the structure of a droplet hosting a singly-quantized linear vortex we have started the imaginary time iteration from a helium density in which the vortex is ``imprinted''. For this purpose, a vortex line along the $z$ can be described by the effective wave function 
			\begin{equation}
				\Psi_0(\mathbf{r}) = \rho_0^{1/2}(r) \, e^{i \, {\cal S}(\mathbf{r})} = \rho_0^{1/2}(\mathbf{r}) \, \frac{(x + i y)}{\sqrt{x^2 + y^2}} \label{eq11}
			\end{equation}
			where $\rho_0(\mathbf{r})$ is the density of either the pure or the impurity-doped droplet without vortex. Vortex lines along other directions passing through a chosen point can be imprinted as well\citep{Pi07}.
			 
			In the case represented by \eq{eq11}, if the impurity is within the vortexcore along a symmetry axis of the impurity-droplet complex,
			the effective wave function $\Psi_0({\mathbf r})$ -- before and after relaxation -- is an eigenvector of the angular 
			momentum operator $\hat{L}_z = -i \; \hbar \partial/\partial \theta$. 
			The angular momentum of the droplet is then
			\begin{equation}
				\langle \hat{L}_z \rangle = \langle \Psi_0(\mathbf{r}) | \hat{L}_z | \Psi_0(\mathbf{r}) \rangle = N \; \hbar
				\label{eq12}
			\end{equation}

	\subsection*{Capture dynamics by vortices}
	To study the interaction of an atomic impurity with vortices, 
we have imprinted a vortex line in the $^4$He$_{1000}$ droplet 
and prepared the Xe atom in different kinematic conditions. 
 
 The inelastic scattering of xenon atoms by quantized vortices in superfluid bulk helium has been addressed in\rf{Psh16}.
 It was found that
 a head-on collision leads to the capture of Xe by the vortex line for $v_0=$ 15.4 m/s, but not for $v_0$=23.7 m/s.
We have carried out an equivalent simulation by initially placing
 the Xe atom inside the droplet 10 \AA{} away
from the vortex line and  sending it head-on towards the vortex at a velocity of 10 m/s.
This velocity is of the order of the thermal velocity of a Xe atom in a droplet under experimental conditions,
once the droplet has thermalized after capturing the Xe atom ($T \sim$0.4 K)\citep{Toe04}. 
Since the equilibrium position of the Xe atom is at the center of the droplet, it moves to this region and remains there during the rest of the simulation.
In this region of the droplet, the Xe atom is also attracted by the vortex, but it is deflected by the superfluid flow around the vortex line and ends up orbiting around it.
 Hence it is captured by the vortex without getting into its core.
 
 	\subsection*{Vortex arrays in droplets}
 	The existence of ordered vortex lattices inside $^4$He droplets has been established by the appearance of Bragg patterns from 
Xe clusters trapped inside the vortex cores in droplets made of $N= 10^8 - 10^{11}$ atoms
(corresponding to radii from 100 to 1000 nm)\citep{Gom14,Jones2016}. We have 
recently studied the stability of vortex 
arrays made of up to $n_v=9$ vortices
inside a $^4$He nanodroplet using the DFT approach\citep{Anc15}. 
It was found that 
the energetically favored structure for $n_v > 6$ is a ring 
of vortices encircling a vortex at the center of the droplet.
Fot $n_v=6$, the 
configuration with a six-vortex ring is found to have almost 
the same energy as the five-fold ring
plus a vortex at the center. The former structure 
has been experimentally observed\citep{Gom14,Jones2016,Ber17}, 
although classical vortex theory 
predicts for it a much higher free energy cost than for the latter\citep{Cam79}.
Similar equilibrium structures have been obtained within DFT for
helium nanocylinders hosting vortex arrays\citep{Anc14}.

We have looked for stationary configurations of a 6-vortex ring
in a rotating He$_{15000}$ droplet by solving the
EL equations in the corotating frame with a fixed
angular velocity. Each vortex core is filled with Ar
atoms, and the system is allowed to fully relax.
In the end, the column of atoms inside each vortex core reaches an equilibrium structure 
where the Ar atoms are separated by a distance which is roughly that of the Ar dimer.
One such configuration is shown in \fig{fig13-capture}. Note that 
the vortex cores are almost straight lines, whereas in an
undoped droplet rotating with the same velocity 
the vortex lines would be bent, 
as shown e.g. in \fig{fig7-capture}.
The Ar atoms are not shown in the Figure.
The localized structures appearing in the vortex cores are 
regions of highly inhomogeneous, high $^4$He density
resulting from the Ar-He attractive potential.
	
	
		We show that Xe and Ar atoms at thermal velocities are readily captured by helium droplets, with a capture cross section similar to the geometric cross section of the droplet. Crucially for the subsequent capture of impurities by vortex lines, we have also shown that most of the kinetic energy of the impinging impurity is lost in the capture process during the first tens of picoseconds. This happens either by the ejection of prompt-emitted He atoms, or by the production of sound waves and large deformations in the droplet. 

		In addition, we have also shown that if the droplet hosts a vortex, slowly moving impurities are readily captured by the vortex line. Rather than being trapped inside the vortex core, the impurity is bound to move at a close distance around it. Besides the crucial energy loss when the impurity hits the droplet, the capture by the vortex is favoured by a further energy transfer from the impurity to the droplet: large amplitude displacements of the vortex line ---as shown in the ESI\citep{ESI}--- take place, constituting another source of the kinetic energy loss in the final stages of the capture. A related issue is the appearance of Kelvin modes in the vortex line, that is not only bent, but also twisted in the course of the collision.

		We can conclude that if the kinematic conditions of the collision (kinetic energy and impact parameter) lead to the capture of the impurity by the droplet, the pinball effect caused by the droplet surface can facilitate the meeting of the Xe/Ar atom and the vortex line ---and the possible capture of the atom by the vortex--- since both have a tendency to remain in the inner region of the droplet. We have shown this in the case of Xe at $v_0$=200~m/s: Xe is captured during its second transit across the droplet, whereas this could not have happened in bulk liquid helium\citep{Psh16}. This effect could explain the capture of impurities by vortex lines even in the very large droplets used in the observation of filament-shaped nanostructures.

	\section*{Perspectives d'avenir}
		Notre travail sur la dynamique en temps réel de la photo-excitation d'un atome de métal alcalin sur la surface d'une nanogouttelette d'hélium est assez vaste. Mener le même type d'études sur d'autres types d'espèces dopantes qui sont solvatées plus profondément à l'intérieur des gouttelettes He (par exemple métaux alcalino-terreux, métaux de transition) permettrait de mieux comprendre les mécanismes de désolvatation et d'éjection des atomes d'impuretés excités. {Loginov: 2007, Loginov: 2012, Kautsch: 2013, Lindebner: 2014}.
		
		De plus, une description plus complète des couplages entre les états électroniques et les degrés de liberté de configuration dans de tels complexes excités induits par l'environnement des gouttelettes He serait hautement souhaitable \ citep {Closser: 2014, Masson: 2014}. Dans une avancée récente, la relaxation électronique des cations Ba $ ^ + $ dans les nanogouttelettes He, basée sur une diabatisation des potentiels d'interaction des états électroniques excités par He-Ba $ ^ + $ et par les états excités \ citep {Vindel: 2018}, a été proposée comme un mécanisme pour éjecter Ba $ ^ + $ et Ba $ ^ + $ He $ _n $ des gouttelettes He. Ces mécanismes de relaxation de spin et de relaxation d'état inter-électronique doivent être confirmés par des études de dynamique en temps réel.
		
		Enfin, les capacités de l'approche He-DFT pourraient aider à élucider des processus d'intérêt expérimental, comme la capture d'une ou plusieurs impuretés par de grosses gouttelettes hébergeant un réseau de vortex et comment plusieurs impuretés atomiques, touchant une gouttelette en rotation hébergeant des vortex, réagissent former de petites grappes, finalement piégés à l'intérieur des noyaux de vortex, comme indiqué par l'apparition de nanostructures en forme de filament dans des expériences.
		
		Dans toutes ces futures lignes de recherche, $ ^ 4 $ He-DFT et TDDFT seront des outils essentiels, étant donné leur capacité à décrire avec précision les propriétés d'équilibre et de dynamique des nanogouttelettes d'hélium de taille réaliste, pouvant accueillir des tourbillons, en interaction avec des dopants. Dans ce contexte, il sera extrêmement intéressant de coupler $ ^ 4 $ He-TDDFT avec la dynamique moléculaire quantique pour aller au-delà de l'approximation du champ moyen de la dynamique du dopant dans l'environnement de l'hélium. Une alternative prometteuse dans l'approche de la fonction de base corrélée (CBF) et son approche multi-composantes récemment développée par Rader \ textit {et al.} \ Citep {Rader2017}