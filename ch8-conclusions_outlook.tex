\chapter{Conclusions \& Outlook}
	\section{Head-on collisions}
		In conclusion, head-on collisions of xenon, a heliophilic atom, involve a kinetic energy exchange of the same order of magnitude as cesium, a heliophobic atom with similar mass.
In both cases, this energy is largely dissipated by  producing energetic waves in the droplet or it is carried away by promptly emitted helium atoms.
The difference is that it takes a much higher velocity for xenon to go through the droplet and escape than for cesium, as could be expected.
Also, density builds up around the xenon during the dynamics, whereas a bubble is created around cesium.

	\section{Capture}
		We have shown
that Xe and Ar atoms at thermal velocities are readily captured by helium droplets, with a capture cross section similar to
 the geometric cross section of the droplet. Crucially for the subsequent capture of impurities by vortex lines,
we have also shown that most of the kinetic energy of the impinging impurity is lost in the capture process during the first tens of picoseconds. This happens either by the ejection of
 prompt-emitted He atoms, or by  the production of sound waves and large deformations in the droplet. 
 
If the droplet hosts a vortex,
slowly moving impurities are readily captured by the vortex line. 
Rather than being trapped inside the vortex core,
 the impurity is bound 
to move at a close distance around it.
Besides the crucial energy loss when the impurity hits the droplet,  the capture by the vortex is favored by a further
energy transfer  from the impurity to the droplet:  large amplitude 
displacements of the vortex line -- as shown in the ESI$\dag$ accompanying this work -- 
take place,  constituting another  source of the kinetic energy loss in the final stages of the capture.
A related issue is 
the appearance of Kelvin modes in the vortex line, that is not only bent,  but also twisted
in the course of the collision.

If the kinematic conditions of the collision (kinetic energy and impact parameter)  lead to the capture of the impurity by the droplet,   the pinball effect
caused by the droplet surface can induce
the meeting 
of the Xe/Ar atom and the vortex line -- and the possible capture of the atom by the vortex -- , since both have a tendency to remain in the inner region of the droplet. 
We have shown this in the case of Xe at $v_0=200$~m/s:
Xe is captured during its second transit across the droplet, whereas this could not have happened in bulk liquid helium.\cite{Psh16}  

The capabilities of the  He-DFT approach might help elucidate 
processes of experimental interest, such
as the capture of  one or several impurities 
by large droplets hosting a vortex  array and
how several atomic impurities, impinging upon a 
rotating droplet hosting vortices,  react 
to form small clusters, eventually being trapped within the 
vortex cores as it appears in the experiments.