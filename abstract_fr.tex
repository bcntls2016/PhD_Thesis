\chapter{Résumé}\markboth{\subsectionfont\itshape\color{activeColor}Résumé}{}
	\lettrine[lines=4]{\color{activeColor}D}{a}ns cette thèse, nous étudions deux aspects de la dynamique d'impuretés atomiques interagissant avec des nanogouttes d'hélium superfluide (He)~: la photo-excitation d'alcalins sur une nanogoutte et le dopage de nanogouttes contenant des tourbillons (vortex) quantiques avec des atomes de gaz rares. Nous utilisons la théorie de la fonctionnelle de la densité d'hélium ainsi que sa version dépendante du temps pour en faire la description théorique.

	Le premier aspect a été effectué dans le cadre d'une collaboration avec des expérimentateurs sur la photo-excitation du rubidium (Rb). Les alcalins sont une sonde très intéressante des gouttelettes d'hélium car ils résident dans leur zone de surface, où il a été prédit qu'un taux de condensation de Bose-Einstein de 100\% était possible en raison d'une densité inférieure à celle de l'hélium superfluide.

	Nos simulations montrent que les états excités 5p et 6p désorbent à des échelles de temps très différentes, séparées par 2 ordres de grandeur ($\sim$100 ps et $\sim$1 ps pour  5p et 6p respectivement). Ces résultats sont en accord avec ceux de l'expérience pompe-sonde à l'échelle femtoseconde qui a étudié la photodesorption d'atomes de Rb. Cependant, dans nos simulations, l'excitation vers $5\mathrm{p}\,^2\Pi_{3/2}$ aboutit à un exciplexe RbHe lié à la surface, contrairement à l'expérience où RbHe est éjecté. L'introduction de la relaxation de spin $^2\Pi_{1/2}\leftarrow{^ 2}\Pi_{3/2}$ nous a permis de résoudre ce désaccord, l'exciplexe RbHe ayant alors assez d'énergie pour désorber. 

	Le deuxième aspect concerne une investigation purement théorique inspirée par les travaux récents de Gomez et Vilesov \emph{et al}., où les tourbillons quantiques étaient visualisés en dopant les nanogouttes d'hélium avec des atomes d'argent, puis en les faisant atterrir en douceur (soft landing) sur un écran de carbone. Les images au microscope électronique montrent de longs filaments d'agrégats d'atomes d'argent qui s'étaient accumulés le long des c\oe urs des vortex. La formation de réseaux de tourbillons quantiques à l'intérieur de nanogoutelettes dopées par du xénon est également mise en évidence par diffraction de rayons X qui montrent des pics de Bragg caractéristiques d'agrégats de xénon piégés dans les c\oe urs des vortex.

	Nous avons d'abord étudié des collisions frontales entre un atome de xénon, héliophile, et une nanogoutte de 1000 héliums, et comparé les résultats à ceux d'une étude précédente sur le même processus avec le césium (Cs), qui est héliophobe. Dans le cas de Xe une «boule de neige»  se forme autour de lui quand il entre dans la nanogoutte, et il lui faut beaucoup plus d'énergie qu'au Cs pour qu'il puisse en ressortir. Quand il le fait, il emporte des héliums avec lui, contrairement au Cs.

	Nous avons ensuite simulé des collisions entre Ar/Xe et des nanogouttes d'hélium superfluides pour différentes vitesses initiales et paramètres d'impact afin de déterminer leur section efficace de capture. Ces simulations ont ensuite été répétées pour des gouttelettes hébergeant un vortex quantique. On observe que l'impact des impuretés induit de grandes déformations de flexion et de torsion de la ligne de vortex, allant jusqu'à la génération d'ondes de Kelvin hélicoïdales qui se propagent le long du c\oe ur du vortex. Ar/Xe est bien finalement capturé par le vortex, mais pas dans son c\oe ur. Nous avons également découvert que l'existence d'un réseau de 6 lignes de vortex dont les noyaux sont remplis d'atomes d'Ar donne une rigidité accrue à la nanogoutte qui permet de stabiliser le système nano-goutte + vortex même à de faibles vitesses angulaires.

	Nos simulations impliquant des nanogouttes d'hélium  comportant des tourbillons quantiques ouvrent la voie à d'autres investigations sur des nanogouttes hébergeant un ensemble de vortex, en collision avec de multiples impuretés.
\clearpage{\pagestyle{empty}\cleardoublepage}